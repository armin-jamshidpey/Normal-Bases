\section{Introduction}

For a finite Galois extension $K/F$ with $G = \mathrm{Gal}(K/F)$, an element $\alpha \in K$
is called normal if the set of Galois conjugates of $\alpha$ forms a basis for $K$ as a
 vector space over $F$. The existence of normal element for any finite Galois extension is 
 known for a long time.
 
There is a wide range of applications of normal bases in finite fields. Fast exponentiation 
or computing the action of the Frobenius and point counting on elliptic curves are some of their applications. There are also 
applications of normal elements in characteristic zero. For a given permutation lattice and appropriate Galois extension
a normal basis is useful in computing the multiplicative invariants explicitly (see \cite{Armin} for more details).   

There are several algorithms for finding a normal element in zero characteristic and finite fields. Due to the applications of
finite fields, normal element in this case is more popular. H. W. Lenstra in \cite{LenstraNormal} introduced a deterministic 
algorithm to construct a normal element which uses $O(n^{O(1)})$ operations over the smaller field where $n$ is the degree of the
 extension. To the best of
our knowledge, the algorithm introduced by Augot and Camion (cite) is the most efficient deterministic one with cost 
$O(n^3+n^2\log q)$
where $q$ is size of the base field. Randomized algorithems for finite fileds are introduced in \cite{Giesbrecht} with cost
$O(n^2+n\log q)$ and \cite{Kaltofen} with cost $O(n^{1.8})$. In characteristic zero, A. Poli gave an algorithm for abelian extensions in \cite{Poli} with $O(n^{O(1)})$ and Schlickewei and Stepanov introduced an algorithm in the cyclic case with
$O(n^{O(1)})$ as the computation cost \cite{Stepanov}. Girstmair has an algorithm which uses $O(n^4)$ operations over the base field to construct a normal element for a general finite Galois extension in characteristic zero \cite{Girstmair}. 

In this paper we will introduce a randomized algorithm for finding a normal element in case of abelian and metacyclic extensions
 which is subquadratic in the degree of the extension. The idea behind the algorithm is similar to ideas of 
\cite{Giesbrecht,Kaltofen}. Since a part of the assumptions in the results of this paper is the same, we state it here 
for future references .

\begin{assumption}\label{assum}
$K/F$ is a finite Galois extension given by $ F[x]/f$ for an irreducible polynomial $f\in F[x]$ of degree $n$, with
 $G = \mathrm{Gal}(K/F)$. Moreover $\alpha$ is an element of $K$ and the action of generators of $ G$, such as $g$, on $K$ is given by its action on $\bar{x} = x \mod f.$ In other words, $g(\bar{x})$ is given for any generator $g$ of $G$.
\end{assumption}

under the Assumption \ref{assum} we choose a random element $\alpha$ of $K$. We use a known fact that, $\alpha$ is normal 
if $M_G(\alpha) \in M_{n\times n}(K)$, an associated matrix to $\alpha$ (see Section 2 for the definition), is invertible. 
Afterwards we reduce the invertiblity of $M_G(\alpha)$ to invertiblity of a random projection an associated element 
$\osum{G}{K} \in K[G]$ which lies in $F[G]$, the group ring of $G$ over $F$. 

Section \ref{sec:pre} of this paper is devoted to provide the definitions and preliminary discussions. In Section \ref{sec:osum} 
the orbit sum problem is discussed and two algorithms are presented which can be applied to compute projections of orbit sums.
Finally in the last section we do what?!

%%% Local Variables:
%%% mode: latex
%%% TeX-master: "NormalBasisCharZero"
%%% End:
