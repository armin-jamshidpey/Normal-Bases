\section{Introduction}

For a finite Galois extension field $\K/\F$, with Galois group $G =
\mathrm{Gal}(\K/\F)$, an element $\alpha \in K$ is called
\emph{normal} if the set of its Galois conjugates $G \cdot \alpha =
\{\sigma\in G: \sigma(\alpha)\}$ forms a basis for $\K$ as a vector space over
$\F$. The existence of normal element for any finite Galois extension
is classical, and constructive proofs are provided in most algebra texts
(see, e.g., \cite{Lang}, Section 6.13).
%\cite{Wae70}, Section~8.11).
 
While there is a wide range of well-known applications of normal bases in
finite fields, such as fast exponentiation~\cite{GaGaPaSh00}, there also
exist applications of normal elements in characteristic zero.  For instance,
in multiplicative invariant theory, for a given permutation lattice and
related Galois extension, a normal basis is useful in computing the
multiplicative invariants explicitly~\cite{Jam18}.

A number of algorithms are available for finding a normal element in
characteristic zero fields and finite fields.  Because of their immediate
applications in finite fields, algorithms for determining normal elements
in this case are most commonly seen.  A fast randomized algorithm for
determining a normal element in a finite field $\FF_{q^n}/\FF_q$, where
$\FF_{q^n}$ is the finite field with $q^n$ elements for any prime power $q$
and integer $n>1$, is presented by \citeN{GatGie90}, with a cost of
$O(n^2+n\log q)$ operations in $\FF_q$.  A faster randomized algorithm is
introduced by \citeN{KalSho98}, with a cost of $O(n^{1.815}\log q)$
operations in $\FF_q$.  In the bit complexity model, Kedlaya and Umans showed
how to reduce the exponent of $n$ to $1.5+\varepsilon$ (for any
$\varepsilon > 0$), by leveraging their quasi-linear time algorithm for
{\em modular composition}~\cite{KeUm11}. \citeN{LenstraNormal} introduced a
deterministic algorithm to construct a normal element which uses $n^{O(1)}$
operations in $\FF_{q^n}/\FF_q$.  To the best of our knowledge, the
algorithm of \citeN{AugCam94} is the most efficient deterministic method,
with a cost of $O(n^3+n^2\log q)$ operations in $\FF_q$.

In characteristic zero, \citeN{SchSte93} gave an algorithm for finding
a normal basis of a number field over $\QQ$ with a cyclic Galois group
of cardinality $n$ which requires $n^{O(1)}$ operations in $\QQ$.
\citeN{Pol94} gives an algorithm for the more general case of finding
a normal basis in an abelian extension $\K/\F$ which requires
$n^{O(1)}$ in $\F$.  More generally in characteristic zero, for any
Galois extension $\K/\F$ of degree $n$ with Galois group given by a
collection of $n$ matrices, \citeN{Girstmair} gives an algorithm which
requires $O(n^4)$ operations in $\F$ to construct a normal element in
$\K$.

In this paper we present a new randomized algorithm for finding a normal
element for abelian and metacyclic extensions, with a runtime subquadratic
in the degree $n$ of the extension. The costs of all algorithms are
measured by counting \emph{arithmetic operations} in $\F$ at unit cost.
Questions related to the bit-complexity of our algorithms are challenging,
and beyond the scope of this paper.

Our main conventions are the following.
\begin{assumption}
  \label{assum}
  Let $\K/\F$ be a finite Galois extension presented as
  $\K=\F[x]/\langle F(x)\rangle$, for an irreducible polynomial $F\in
  \F[x]$ of degree $n$. Then,
  \begin{itemize}
  \item elements of $\K$ are written on the power basis $1,\xbar,\dots,\xbar^{n-1}$,
    where $\xbar = x \bmod F$;
  \item elements of $G$ are represented by their action on $\xbar$.
  \end{itemize}
\end{assumption}

In particular, for $g \in G$ given by means of $\gamma:=g(\xbar) \in \K$,
and $\beta = \sum_{0\leq i<n}\beta_i\xbar^i\in\K$, the fact that $g$ is an
$\F$-automorphism implies that $g(\beta)$ is equal to $\beta(\gamma)$, the
polynomial composition of $\beta$ at $\gamma$ (reduced modulo $F$).

Our algorithms combine techniques and ideas due
to~\cite{GatGie90,KalSho98}: $\alpha \in \K$ is normal if and only
if the element $S_\alpha := \sum_{\sigma \in G} \sigma(\alpha)\sigma
\in \K[G]$ is invertible in the group algebra $\K[G]$. The algorithms
choose $\alpha$ at random; a generic choice is normal (so we expect
$O(1)$ random trials to be sufficient). However, writing down
$S_\alpha$ involves $\Theta(n^2)$ elements in $\F$, which precludes a
subquadratic runtime. Knowing $\alpha$, the algorithms use a
randomized reduction to a similar question in $\F[G]$, that amounts to
applying a random projection $\K\to\F$ to all entries of $S_\alpha$.
For that, we adapt algorithms from~\cite{KalSho98} that were written
for Galois groups of finite fields.

Section \ref{sec:pre} of this paper is devoted to definitions and
preliminary discussions.  In Section \ref{sec:osum} two subquadratic-time
algorithms are presented for the randomize reduction of our main question
to invertibility testing in $\F[G]$, for respectively abelian and
metacyclic groups.  Finally, in Section \ref{sec:invertibility}, we show
that the latter problem can be solved in subquadratic time as well for the
families of groups we consider.

This paper is written from the point of view of obtaining improved
asymptotic complexity estimates. Since our main goal is to highlight the
exponent (in $n$) in our runtime analyses, costs are given using the soft-O
notation: $S(n)$ is in $\tilde{O}(T(n))$ if it is in
$O(T(n) \log(T(n))^c)$, for some constant $c$. Our algorithms make
extensive of known algorithms for polynomial and matrix arithmetic; in
particular, we use repeatedly the fact that polynomials of degree $d$ in
$\F[x]$, for any field $\F$ of characteristic zero, can be multiplied in
$\tilde{O}(n)$ operations in $\F$~\cite{ScSt71}. As a result, arithmetic 
operations $(+,\times,\div)$ in $\K$ can all be done using $\tilde{O}(n)$ 
operations in $\F$.

For matrix arithmetic, we will rely on some non-trivial results on
rectangular matrix multiplication initiated by \citeN{LoRo83}. For $k \in
\mathbb{R}$, we denote by $\omega(k)$ a constant such that over any
ring, matrices of sizes $(n,n)$ by $(n,\lceil n^k \rceil)$ can be
multiplied in $O(n^{\omega(k)})$ ring operations (so $\omega(1)$ is
the usual exponent of square matrix multiplication, which we simply
write $\omega$).  The sharpest values known to date for most
rectangular formats are from~\cite{LeGall}; for $k=1$, the best known
value is $\omega \le 2.373$ by \citeN{LeGall14}. Over a field, we will
frequently use the fact that further matrix operations (determinant or
inverse) can be done in $O(n^\omega)$ base field operations.


%%% Local Variables:
%%% mode: latex
%%% TeX-master: "NormalBasisCharZero"
%%% End:
