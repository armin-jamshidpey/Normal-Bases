\section{Complexity of Arithmetic Operations in the Group Algebra}

The problems we consider in this section are of independent interest:
given an element $\alpha$ in $\F[G]$, for a field $\F$ and a group $G$,
determine whether $\alpha$ is a unit in $\F[G]$. 

Since we are in characteristic zero, Wedderburn's theorem implies the
existence of an $\\F$-algebra isomorphism
$$\varphi: \F[G] \to M_{d_1}(D_1) \times \dots \times M_{d_r}(D_r),$$
where all $D_i$'s are division algebras over $\F$. If we were working
over $\F=\C$, all $D_i$'s would simply be $\C$ itself; then, a natural
solution to test the invertibility of $\alpha$ would be to compute its
Fourier transform $\varphi(\alpha)$ and test whether all its
components $\varphi_1(\alpha) \in M_{d_1}(\C),\dots,\varphi_r(\alpha)
\in M_{d_r}(\C)$ are invertible. This boils down to linear algebra
over $\C$, and takes $O(d_1^\omega + \cdots + d_r^\omega)$ operations.
Since $d_1^2 + \cdots + d_r^2 = |G|$, this is $O(|G|^{\omega/2})$
operations in $\C$.

However, we do not wish to make such an assumption as $\F=\C$. Since we
measure the cost of our algorithms in $\F$-operations, the direct
approach that embeds $\F[G]$ into $\C[G]$ does not make it possible to
obtain a subquadratic cost in general. If for instance $\F=\Q$ and $G$
is cyclic of order $n=2^k$, $\varphi$ is the usual Cooley-Tukey
Fourier transform, but computing it requires we work in a degree $n/2$
extension of $\Q$, whence a quadratic runtime.

On the other hand, the Wedderburn decomposition over $\F$ may not be
trivial to compute. As a result, for the families of groups we have
considered so far, we will work with an 
 $\\F$-algebra isomorphism
$$\psi: \F[G] \to M_{c_1}(R_1) \times \dots \times M_{c_s}(R_s),$$
bla bla bla.


%%%%%%%%%%%%%%%%%%%%%%%%%%%%%%%%%%%%%%%%%%%%%%%%%%%%%%%%%%%%

\subsection{Abelian groups}

Because an abelian group is a product of cyclic groups, the group
algebra $\F[G]$ of such a group is the tensor product of algebras of the
form $\F[x_i]/\langle {x_i}^{n_i}-1 \rangle$. Given an element $A$ in
$\F[G]$, seen as a multivariate polynomial, our problem in
this section is to determine whether $A$ is a unit.

The complexity of arithmetic operations in an $\F$-algebra such as
$\A:=\F[x_1,\dots,x_t]/\langle P_1(x_1),\dots,P_t(x_t)\rangle$ is
difficult to pin down exactly. For general $P_i$'s, the cost of
multiplication in $\A$ is known to be $O(\dim(\A)^{1+\varepsilon})$, for
any $\varepsilon > 0$~\cite[Theorem~2]{LiMoSc09}; from this, it may be
possible to deduce similar upper bounds on the complexity of invertibility 
tests, following~\cite{DaMMMScXi06}, but this seems non-trivial.

Instead, we give an algorithm with softly linear runtime, that uses
the factorization properties of cyclotomic polynomials and Chinese
remaindering techniques to turn our problem into that of invertibility
testing in algebras of the form $\F[x]/P$, for various polynomials
$P$.

\smallskip

\noindent{\bf Tensor product of two cyclotomic rings: coprime orders.}
Let $m,m'$ be two coprime integers and define
$$\mathbbm{h}:=\F[x,x']/\langle \Phi_{m}(x), \Phi_{m'}(x')\rangle,$$
where for $i \ge 0$, $\Phi_i$ is the cyclotomic polynomial of order
$i$. In what follows, $\varphi$ is Euler's totient function, so that
$\varphi(i) = \deg(\Phi_i)$ for all~$i$.
\begin{lemma}
  There exist an $\F$-algebra isomorphism $\gamma: \mathbbm{h} \to
  \F[z]/\langle\Phi_{mm'}(z)\rangle$ given by $xx' \mapsto z$.  Given
  $\Phi_m$ and $\Phi_{m'}$, $\Phi_{mm'}$ can be computed in time
  $\tilde{O}(\varphi(mm'))$; knowing all these polynomials, one can
  apply $\gamma$ and its inverse to any input in time
  $\tilde{O}(\varphi(mm'))$.
\end{lemma}
\begin{proof}
  Without loss of generality, we prove the first claim over $\Q$; the
  result over $\F$ follows by scalar extension. In the field \sloppy
  $\Q[x,x']/\langle \Phi_{m}(x), \Phi_{m'}(x')\rangle$, $xx'$ is
  cancelled by $\Phi_{mm'}$. Since this polynomial is irreducible, it
  is the minimal polynomial of $xx'$, which is thus a primitive
  element for $\Q[x,x']/\langle \Phi_{m}(x),
  \Phi_{m'}(x')\rangle$. This proves the first claim.

  For the second claim, we first determine the images of $x$ and $x'$
  by $\gamma$. Start from a B\'ezout relation $am+ a'm'=1$, for some
  $a,a'$ in $\Z$.  Since $x^m = {x'}^{m'}=1$ in $\mathbbm{h}$, we
  deduce that $\gamma(x)=z^{u}$ and $\gamma(x') = z^{v}$, with $u:=am
  \bmod mm'$ and $v:=a'm' \bmod mm'$. To compute $\gamma(P)$, for some
  $P$ in $\mathbbm{h}$, we first compute $P(z^u, z^v)$, keeping all
  exponents reduced modulo $mm'$. This requires no arithmetic
  operations and results in a polynomial $\bar P$ of degree less than
  $mm'$, which we eventually reduce modulo $\Phi_{mm'}$ (the latter is
  obtained by the composed product algorithm of~\cite{BoFlSaSc06} in
  quasi-linear time).  By~\cite[Theorem~8.8.7]{BaSh96}, we have the
  bound $s = O(\varphi(s) \log(\log(s)))$, so that $s$ is in
  $\tilde{O}(\varphi(s))$. Thus, we can reduce $\bar P$ modulo
  $\Phi_{mm'}$ in $\tilde{O}(\varphi(mm'))$ operations, establishing
  the cost bound for $\gamma$.

Conversely, given $Q$ in $\F[z]/\langle\Phi_{mm'}(z)\rangle$, we
obtain its preimage by replacing powers of $z$ by powers of $xx'$,
reducing all exponents in $x$ modulo $m$, and all exponents in $x'$
modulo $m'$; then, we reduce the result modulo both $\Phi_m(x)$ and
$\Phi_{m'}(x')$.  By the same argument as above, the cost is softly
linear in $\varphi(mm')$.
\end{proof}

\noindent{\bf Extension to several cyclotomic rings.}  The
natural generalization of the algorithm above starts with pairwise
distinct primes $p_1,\dots,p_t$, non-negative exponent $c_1,\dots,c_t$
and variables $x_1,\dots,x_t$ over $\F$. Now, we define
$$\H:=\F[x_1,\dots,x_t]/\langle
\Phi_{\mu_1}(x_1),\dots,\Phi_{\mu_t}(x_t)\rangle,$$ with
$\mu_i={p_i}^{c_i}$ for all $i$; when needed, we will write
$\H_{p_1,c_1,\dots,p_t,c_t}$. Finally, we let  $\mu:=\mu_1\cdots \mu_t$;
then, the dimension $\dim(\H)$ is $\varphi(\mu)$

\begin{lemma}
 There exist an $\F$-algebra isomorphism $\Gamma: \H \to
 \F[z]/\langle\Phi_{\mu}(z)\rangle$ given by $x_1 \cdots x_t \mapsto
 z$.  One can apply $\Gamma$ and its inverse to any input in time
 $\tilde{O}(\dim(\H))$.
\end{lemma}
\begin{proof}
  We proceed iteratively. First, note that the cyclotomic polynomials
  $\Phi_{\mu_i}$ can all be computed in time $O(\varphi(\mu))$. 
  Then, the isomorphism
  $\gamma: \F[x_1,x_2]/\langle \Phi_{\mu_1}(x_1),
  \Phi_{\mu_2}(x_2)\rangle \to \F[z]/\langle
  \Phi_{\mu_1\mu_2}(z)\rangle$
given in the previous paragraph extends coordinate-wise to an
  isomorphim
  $$\Gamma_1: \H \to \F[z,x_3,\dots,x_t]/\langle
  \Phi_{\mu_1\mu_2}(z),\Phi_{\mu_3}(x_3),\dots,\Phi_{\mu_t}(x_t)\rangle.$$
  By the previous lemma, $\Gamma_1$ and its inverse can be applied to
  any input in time $\tilde{O}(\varphi(\mu))$. Iterate this process
  another $t-2$ times, to obtain $\Gamma$ as a product
  $\Gamma_{t-1} \circ \cdots \circ \Gamma_1$. Since $t$ is logarithmic 
  in $\varphi(\mu)$, the proof is complete.
\end{proof}

\noindent{\bf Tensor product of two cyclotomic rings with same $p$.}
In the following two paragraphs, we discuss the opposite situation as
above: we now work with cyclotomic polynomials of prime power
orders for a common prime $p$. As above, we start with two such polynomials.

Let thus $p$ be a prime. The key to the following algorithms is the
elementary remark below.  Let $c,c'$ be positive integers, with $c \ge
c'$, and let $y$ be another indeterminate over $\F$. Define
$\mathbbm{a}:=\F[x]/\Phi_{p^c}(x)$, and let $\bar x$ be the residue class of
$x$ in $\A$. Then, in $\mathbbm{a}[y]$, $\Phi_{p^{c'}}(y)$ factors as
$$\Phi_{p^{c'}}(y) =\prod_{\substack{1 \le i\le p^{c'}-1\\ \gcd(i,p) =1}}
(y-\rho_i),$$ with $\rho_i:={\bar x}^{i p^{c-c'}}$ for all $i$.  Even
though $\mathbbm{a}$ may not be a field, the Chinese Remainder theorem
implies that the algebra
$$\mathbbm{b}:=\F[x,y]/\langle \Phi_{p^c}(x), \Phi_{p^{c'}}(y)\rangle = \mathbbm{a}[y]/\Phi_{p^{c'}}(y)$$
is isomorphic to $\mathbbm{a}^{\varphi(p^{c'})}$;
the isomorphism is given by
$$\begin{array}{cccc}
\theta: & \mathbbm{b} & \to & \mathbbm{a} \times \cdots \times \mathbbm{a} \\
        & P & \mapsto& (P(\bar x,\rho_1),\dots,P(\bar x,\rho_{\varphi(p^{c'})}).
\end{array}$$
In terms of complexity, arithmetic operations $(+,-,\times)$ in $\mathbbm{a}$
can all be done in $\tilde{O}(\varphi(p^c))$ operations in $\F$.
Starting from $\rho_1 \in \mathbbm{a}$, all other roots $\rho_i$ can be
computed in $O(\varphi(p^{c'}))$ operations in $\mathbbm{a}$, that is,
$\tilde{O}(\varphi(p^c) \varphi(p^{c'}))$ operations in $\mathbbm{b}$.  We
rewrite this as $\tilde{O}(\deg(\mathbbm{b}))$, where $\dim(\mathbbm{b}):=\varphi(p^c)
\varphi(p^{c'})$ is the dimension of $\mathbbm{b}$ as an $\F$-vector space.

Applying $\theta$ and its inverse is done by means of fast evaluation
and interpolation~\cite[Chapter~10]{vzGathen13} in $\tilde{O}(\varphi(p^{c'}))$
operations in $\mathbbm{a}$, that is, $\tilde{O}(\deg(\mathbbm{b}))$ operations in $\F$
(the algorithms do not require that $\mathbbm{a}$ be a field).

\smallskip\noindent{\bf Extension to several cyclotomic algebras.}
Let $p$ be as before, and consider now positive integers
$c_1,\dots,c_t$ and variables $x_1,\dots,x_t$. We define the
$\F$-algebra
$$\A:=\F[x_1,\dots,x_t]/\langle \Phi_{p^{c_1}}(x_1), \dots,
\Phi_{p^{c_t}}(x_t)\rangle,$$ which we will sometimes write
$\A_{p,c_1,\dots,c_t}$ to make the dependency on $p$ and the $c_i$'s
clear. Up to reordering the $c_i$'s, we can assume that $c_1 \ge c_i$
holds for all $i$, and define as before
$\mathbbm{a}:=\F[x_1]/\Phi_{p^{c_1}}(x_1)$.

\begin{lemma}\label{lemma:A}
  There exists an $\F$-algebra isomorphism $\Theta: \A \to
  \mathbbm{a}^{\dim(\A)/\dim(\mathbbm{a})}$. This isomorphism and its
  inverse can be applied to any inputs in $\tilde{O}(\dim(\A))$
  operations in $\F$.
\end{lemma}
\begin{proof}
We proceed iteratively. First, rewrite $\A$ as
$$\A=\mathbbm{a}[x_2,x_3,\dots,x_t]/\langle \Phi_{p^{c_2}}(x_2), \Phi_{p^{c_3}}(x_3), \dots,
F_{c_t}(x_t)\rangle.$$ 
The isomorphism 
$\theta: \mathbbm{a}[x_2]/\Phi_{p^{c_2}}(x_2) \to \mathbbm{a}^{\varphi(p^{c_2})}$
introduced in the previous paragraph extends coordinate-wise
to an isomorphism 
$$\Theta_1: \A \to (\mathbbm{a}[x_3,\dots,x_t]/\langle \Phi_{p^{c_3}}(x_3),
\dots, \Phi_{p^{c_t}}(x_t)\rangle)^{\varphi(p^{c_2})};$$ $\Theta_1$ and its
inverse can be evaluated in quasi-linear time $\tilde{O}(\dim(\A))$.
We now work in all copies of $\mathbbm{a}[x_3,\dots,x_t]/\langle
\Phi_{p^{c_3}}(x_3), \dots, \Phi_{p^{c_t}}(x_t)\rangle$ independently, and apply the
procedure above to each of them. Altogether, we have $t-1$ such 
steps to perform, giving us an isomorphism 
$$\Theta = \Theta_{t-1} \circ \cdots \circ \Theta_1:
\A \to
\mathbbm{a}^{\varphi(p^{c_2}) \cdots \varphi(p^{c_t})}.$$
The exponent can be rewritten as $ \dim(\A)/\dim(\mathbbm{a})$, as claimed.
In terms of complexity, all $\Theta_i$'s and their inverses can be computed
in quasi-linear time $\tilde{O}(\dim(\A))$, and we do $t-1$ of them,
where $t$ is $O(\log(\dim(\A)))$. The proof is complete.
\end{proof}

\noindent{\bf Decomposing certain $p$-group algebras.}  The prime $p$
and indeterminates $x_1,\dots,x_t$ are as before; we now consider
positive integers $b_1 \ge \cdots \ge b_t$, and the $\F$-algebra
$$
\begin{array}{ccc}
\B&:=&\F[x_1,\dots,x_t]/\langle x_1^{p^{b_1}}-1,\dots,x_t^{p^{b_t}}-1\rangle\\$$
&=& \F[x_1]/\langle x_1^{p^{b_1}}-1 \rangle \otimes \cdots \otimes \F[x_t]/\langle x_t^{p^{b_t}}-1 \rangle;
\end{array}$$
if needed, we will write $\B_{p,b_1,\dots,b_t}$ to make the dependency
on $p$ and the $b_i$'s clear. This is the $\F$-group algebra
of $\Z/p^{b_1}\Z \times \cdots \times \Z/p^{b_t}\Z$.

For $i \le t$, we have the  factorization 
$$x_i^{p^{b_i}}-1 = (x_i-1) \Phi_p(x_i)
\Phi_{p^2}(x_i) \cdots \Phi_{p^{b_i}}(x_i).$$ The factors may not be irreducible,
but they are pairwise coprime, so that we have a Chinese Remainder
isomorphism
$$\lambda_i: \F[x_i]/\langle x_i^{p^{b_i}}-1 \rangle \to \F[x_i]/\langle x_i-1\rangle
\times \cdots \times  \F[x_i]/\langle \Phi_{p^{b_i}}(x_i)\rangle;$$
together with its inverse, it can be computed  
in $\tilde{O}(p^{b_i})$ operations in $\F$~\cite[Chapter~10]{vzGathen13}. By distributivity of the tensor
product over direct products, 
this gives an $\F$-algebra isomorphism
$$\lambda: \B \to \prod_{c_1=0}^{b_i} \cdots \prod_{c_t=0}^{b_t} \A_{p,\boldsymbol{c}'};$$
in this expression, $\boldsymbol{c}'$ is obtained from
$\boldsymbol{c}=(c_1,\dots,c_t)$ by removing all $c_i$ equal to $0$;
they correspond to the factor $x_i-1$, and $\F[x_i]/\langle x_i-1\rangle$
is simply equal to $\F$. Together with its inverse, 
$\lambda$ can be computed in $\tilde{O}(\dim(\B))$ operations in $\F$.
Composing with the result in Lemma~\ref{lemma:A}, this gives
us an isomorphim
$$\Lambda: \B \to \B':=\prod_{c_1=-1}^{b_i-1} \cdots \prod_{c_t=-1}^{b_t-1}
\mathbbm{a}_{c_1,\dots,c_t}^{D_{c_1,\dots,c_t}},$$ where
$\mathbbm{a}_{c_1,\dots,c_t} = \F[x]/F_{p^c}$, with $c
=\max(c_1,\dots,c_t)$ and $D_{c_1,\dots,c_t} =
\dim(\A_{t,c_1,\dots,c_t})/\dim(\mathbbm{a}_{c_1,\dots,c_t})$.  As
before, $\Lambda$ and its inverse can be computed in quasi-linear time
$\tilde{O}(\dim(\B))$. As for $\B$, we will 
write $\B'_{p,b_1,\dots,b_t}$ if needed (it is well-defined, up to the order of the factors).

\noindent{\bf Main result.}  As in Subsection~\ref{ssec:proj_abelian}, suppose
that $G$ is an abelian group, presented as
$$ \langle g_1, \ldots , g_r: g_{1}^{e_1} = \cdots = g_{r}^{e_r} = 1
\rangle$$ where $ e_i \in \mathbb{N}$ is the order of $g_i$ and $n =
e_1 \cdots e_r$. 

Let us write the elementary divisor decomposition of $G$ as $G = G_1
\times G_s$, where each $G_i$ is of prime power order $p_i^{a_i}$, so
that $n = p_1^{a_1} \cdots p_s^{a_s}$, where $p_1 < \cdots < p_s$ are
primes ($G_i$ is the $p_i$-Sylow subgroup of $G$). Each $G_i$ can
itself be written as a product of cyclic groups, $G_i = G_{i,1} \times
\cdots \times G_{i,t_i}$, where the factor $G_{i,j}$ is cyclic of
order ${p_i}^{b_{i,j}}$, with $b_{i,1} \le \cdots \le b_{i,t_i}$; this
is the invariant factor decomposition of $G_i$, with $b_{i,1} + \cdots
+ b_{i,t_i} = a_i$.

We will henceforth assume that generators
$\gamma_{1,1},\dots,\gamma_{s,t_s}$ of respectively
$G_{1,1},\dots,G_{s,t_s}$ are known, and that elements of $\F[G]$ are
given on the power basis in $\gamma_{1,1},\dots,\gamma_{s,t_s}$. Were
it not case, a brute-force solution would factor each $e_i$ (factoring
$e_i$ takes $o(e_i)$ bit operations on a standard RAM), so as to write
$\langle g_i \rangle$ as a product of cyclic groups of prime power
orders, from which the required decomposition follows.

From the factorization $G = G_1 \times \cdots \times G_s$, we deduce
that the group algebra $\F[G]$ is the tensor product $\F[G_1]
\otimes \cdots \otimes \F[G_s]$ (taken over $\F$). Further, the 
factorization $G_i = G_{i,1} \times \cdots \times G_{i,t_i}$
implies that $\F[G_i]$ is isomorphic, as an $\F$-algebra, to
$$\F[x_{1},\dots,x_{t_i}]/\left \langle x_{1}^{p_i^{b_{1}}}-1,\dots,x_{1}^{p_i^{b_{i,t_i}}}-1\right\rangle
=\B_{p,b_{i,1},\dots,b_{i,t_i}}.$$


Given $\alpha = \sum \alpha_{i_1,\dots,i_r} g_1^{i_1}\cdots
g_r^{i_r}$, its image in $\A$ is simply 
$$A= \sum
\alpha_{i_1,\dots,i_r} x_1^{i_1}\cdots x_r^{i_r}.$$
{\bf todo: apply distributivity tensor product to get an isomorphism
$\F[G]$ to a product of $\B'$'s}





\subsection{Metacyclic case}


\begin{lemma}\label{lem:metinjection}
Suppose $G = \langle \sigma , \tau : \sigma^n = 1, \tau^m = \sigma^t, \tau^{-1} \sigma \tau = \sigma^r \rangle$ is the Galois group 
of $K/F$, where $n$ is a prime integer. The map $\varphi = \varphi_1 \oplus \varphi_2$ where,
\begin{equation}
\begin{split}
\varphi_1: K[G] \longrightarrow M_{m}(F(\zeta))\\
\sigma \longrightarrow \mathrm{Diag}(\zeta, \zeta^r, \ldots , \zeta^{r^{m-1}})\\
\tau \longrightarrow 
\left[ \begin{array}{l|l}
0 & 1\\
\hline
I_{m-1}& 0
\end{array} \right] \\
\varphi_2: K[G] \longrightarrow M_m(F) \\
\sigma \longrightarrow I_m \\
\tau \longrightarrow 
\left[ \begin{array}{l|l}
0 & 1\\
\hline
I_{m-1}& 0
\end{array} \right] 
\end{split}
\end{equation}
where $\zeta$ is a primitive $n$-th root of unity, is an injective homomorphism.
\end{lemma}

\begin{proof}
It is not hard to verify that 
$$\varphi(\sigma)^n = I_m$$
$$\varphi(\tau)^m = \varphi(\sigma)^t$$
$$\varphi(\tau)^{-1}\varphi_i(\sigma) \varphi_i(\tau) = \varphi_i(\sigma)^r$$
which shows that $\varphi$ is a well defined homomorphism. 

Assume $u = \sum_{j = 0}^m \left( \sum_{i = 0}^n u_{ij} \sigma^i \right) \tau^j \in \mathrm{Ker}(\varphi).$ If
$F_j(x) = \sum_{i = 0}^n u_{ij}x^i$, $F_{i,j} = F_i(\zeta^{r^j})$ and $\bar{F_i} = F_i(1)$, then 
$$\varphi_1(u) = 
\left[\begin{array}{lllll}
F_{0,0} & F_{m-1,1}	& \cdots	&	F_{2,0} & F_{1,0}\\
F_{1,1} & F_{0,1}& F_{m-1,2} &\cdots & F_{2,1}\\
F_{2,3} & F_{1,2}& F_{0,2} &\cdots & F_{3,m-1}\\
&\ddots & \ddots & \ddots&\\
F_{m-1,0}	&	&	& F_{1,m-1}	& F_{0,m-1}
\end{array}
\right].
$$
Hence we get $F_i(\zeta^{r^j}) =0$. On the other hand by applying $\varphi_2$ we get
$$\varphi_1(u) = 
\left[\begin{array}{lllll}
\bar{F}_{0} & \bar{F}_{m-1}	& \cdots	&	\bar{F}_{2} & \bar{F}_{1}\\
\bar{F}_{1} & \bar{F}_{0}& \bar{F}_{m-1} &\cdots & \bar{F}_{2}\\
\bar{F}_{2} & \bar{F}_{1}& \bar{F}_{0} &\cdots & \bar{F}_{3}\\
&\ddots & \ddots & \ddots&\\
\bar{F}_{m-1}	&	&	& \bar{F}_{1}	& \bar{F}_{0}
\end{array}
\right].
$$
$F_i(1) = 0$. Hence
$F_i = 0$ for $0 \leq i \leq m-1$, which means that $\varphi$ is injective.
\end{proof}

\begin{proposition}
Under Assumption \ref{assum}, let $G$ be a metacyclic group presented in Lemma \ref{lem:metinjection}. The invertibility of $l(\osum{G}{K}) \in F[G]$ can be decided using $O(m^2n)$ operations in $F$, for a linear projection $l: K\rightarrow F$.
\end{proposition}



%%% Local Variables:
%%% mode: latex
%%% TeX-master: "NormalBasisCharZero"
%%% End:
