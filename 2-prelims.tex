\section{Preliminaries}
\label{sec:pre}

One of the well-known proofs of the existence of a normal element for a
finite Galois extension \cite[Theorem 6.13.1]{Lang} suggests a randomized
algorithm for finding such an element. Assume $\K/\F$ is a finite Galois
extension with Galois group $G = \lbrace g_1 , \ldots , g_n \rbrace$. If
$x \in \K$ is a normal element, then
\begin{equation}
  \label{eq:fstrow}
  \sum_{j=1}^n 
  c_j g_j(x)=0, \,\,\, c_j \in \F 
\end{equation} 
implies $c_1 =\ldots=c_n = 0$. For each
$i \in \lbrace 1, \ldots , n\rbrace$, applying $g_i$ to equation
\eqref{eq:fstrow} yields
\begin{equation} \label{eq:otherrow} \sum_{j=1}^n c_j g_i g_j(x)=0.
\end{equation}
Using \eqref{eq:fstrow} and \eqref{eq:otherrow}, one can form a linear
system $\mat{M}_G(\alpha)\textbf{v} = \textbf{0}$ where, for $\alpha\in\K$,
\[
  \mat M_G(\alpha) =
  \begin{bmatrix}
    g_1 g_1(\alpha) & g_1 g_2(\alpha) & \cdots & g_1 g_n(\alpha) \\
    g_2 g_1(\alpha) & g_2 g_2(\alpha) & \cdots & g_2 g_n(\alpha) \\
    \vdots		& \vdots	& \vdots & \vdots \\
    g_n g_1(\alpha) & g_n g_2(\alpha) & \cdots & g_n g_n(\alpha) \\
  \end{bmatrix} \in M_n(\K).
\]
Classical proofs then proceed to show that there exists $\alpha \in \K$
with $\det(\mat M_G(\alpha))\neq 0$.
 
This approach can be used as the basis of a randomized algorithm for
finding a normal element: choose a random element $\alpha$ in $\K$ until we
find one such that $ \mat M_G(\alpha)$. A direct implementation computes
all the entries of the matrix and then use linear algebra to compute its
determinant; using fast matrix arithmetic this requires $O(n^\omega)$
operations in $\K$, that is $\tilde{O}(n^{\omega+1})$ operations in
$\F$. This is at least cubic in $n$, and only a minor improvement over the
previously best-known approach of \citeN{Girstmair}. The main contribution
of this paper is to show how to speed up this verification.
 
Before entering that discussion, we briefly discuss the probability that
$\alpha$ be a normal element: if we write
$\alpha = a_0 + \cdots + a_{n-1} \bar x^{n-1}$, the determinant of
$\mat M_G(\alpha)$ is a (not identically zero) homogeneous polynomial of
degree $n$ in $(a_0,\dots,a_{n-1})$. If the $a_i$'s are chosen uniformly at
random in a finite set $X \subset \F$, the Lipton-DeMillo-Schwartz-Zippel
implies that the probability that $\alpha$ be normal is at leat $1-n/|X|$.

If $G$ is cyclic, \cite{GatGie90} avoid the determinant computation by
computing the GCD of $S_\alpha(x) := \sum_{i = 0}^{n-1} g_i(\alpha)x^i$ and
$x^n-1$. In effect, this amounts to testing whether $S_\alpha$ is
invertible in the group ring $\K[G]$, which is isomorphic to
$\K[x]/(x^n-1)$. This is a general fact: for any $G$, $\mat M_G(\alpha)$ is
the matrix of (left) multiplication by the orbit sum
$$S_\alpha:= \sum_{g \in G} g(\alpha)g \in \K[G],$$ and $\alpha$ being
normal is equivalent to $S_\alpha$ being a unit in $\K[G]$. This point of
view may make it possible to avoid linear algebra of size $n$ over $\K$,
but writing $S_\alpha$ itself still involves $\Theta(n^2)$ elements in
$\F$. The following lemma is the main new ingredient in our algorithm: it
gives a randomized reduction to testing whether a suitable projection of
$S_\alpha$ in $\F[G]$ is a unit.
 
%% This gives an idea to modify Algorithm \ref{Alg:Naive}. Instead of
%% writing $M_G(x)$ and test its invertiblity, we can write $\osum{G}{\K}$
%% and test if it is invertible in $\K[G]$. Although testing the
%% invertibility of $\osum{G}{\K}$ might be efficient in comparison to
%% computing the determinant of a matrix in $\K$, we prefer to do the
%% computations over $\F$ rather than $\K$. The following lemma comes handy
%% to pass the computations from $\K$ to $\F$.


\begin{lemma}
  \label{Lem:Proj}
  For $\alpha \in \K$, $\mat M_G(\alpha)$ is invertible if and only
  if $$\ell(\mat M_G(\alpha)) := [\ell(g_ig_j(\alpha))]_{ij} \in M_n(\F)$$
  is invertible, for a generic $\F$-linear projection $\ell: \K \to \F$.
\end{lemma}
\begin{proof}
  $(\Rightarrow)$ For a fixed $\alpha\in\K$, any entry of
  $\mat M_G(\alpha)$ can be written as
  \begin{equation}\label{Eq:PrimElm}
    \sum_{k= 0}^{n-1} a_{ijk}\bar x^k,
  \end{equation}
  and the corresponding entry in $\ell(\mat M_G(\alpha))$, for
  $\ell: \K \to \F$ can be written $\sum_{k= 0}^{n-1} a_{ijk}\ell_k$, with
  $\ell_k = \ell(\bar x^k)$. Replacing these $\ell_k$'s by indeterminates
  $L_k$'s, the determinant becomes a polynomial in
  $P \in \F[L_1, \ldots, L_n].$ Viewing $P$ in $\K[x_1, \ldots , x_n]$, we
  have $ P(1, \bar x, \ldots, \bar x^{n-1})$ $= \det(\mat M_G(\alpha))$,
  which is non-zero by assumption. Hence, $P$ is not identically zero, and
  the conclusion follows.
  
  $(\Leftarrow)$ Assume $\mat M_G(\alpha)$ is not invertible. Following the
  proof of \cite[Lemma 4]{Jam18}, we first show that there exists a
  non-zero $\boldsymbol{u} \in \F^n$ in the kernel of $\mat M_G(\alpha)$.
  
  The elements of $G$ act on rows of $\mat M_G(\alpha)$ entrywise and the
  action permutes the rows the matrix. Assume
  $\varphi : G \to \mathfrak{S}_n$ is the group homomorphism such that
  $g(\mat M_i) = \mat M_{\varphi(g)(i)}$ for all $i$, where $\mat M_i$ is
  the $i$-th row of $\mat M_G(\alpha)$.
  
  Since $\mat M_G(\alpha)$ is singular, there exists a non-zero
  $\boldsymbol{v} \in \K^n$ such that $\mat M_G(\alpha)\boldsymbol{v} = 0$;
  we choose $\boldsymbol{v}$ having the minimum number of non-zero
  entries. Let $i \in \lbrace 1, \ldots , n \rbrace$ such that
  $v_i \neq 0$. Define $\boldsymbol{u} = 1/v_i\boldsymbol{v}$. Then,
  $\mat M_G(\alpha)\boldsymbol{u} = 0,$ which means
  $\mat M_j \boldsymbol{u} = 0 $ for $j \in \lbrace 1, \ldots, n
  \rbrace$. For $g \in G$, we have
  $g(\mat M_j \boldsymbol{u}) = \mat M_{\varphi(g)(j)} g(\boldsymbol{u})=
  0.$ Since this holds for any $j$, we conclude that
  $\mat M_G(\alpha)g(\boldsymbol{u})= 0$, hence
  $g(\boldsymbol{u})-\boldsymbol{u}$ is in the kernel of
  $\mat M_G(\alpha)$. On the other hand since the $i$-th entry of
  $\boldsymbol{u}$ is one, the $i$-th entry of
  $g(\boldsymbol{u}) -\boldsymbol{u}$ is zero. Thus the minimality
  assumption on $\textbf{v}$ shows that
  $g(\boldsymbol{u}) -\boldsymbol{u} = 0$ and equivalently
  $g(\boldsymbol{u})=\boldsymbol{u}$ and hence $\boldsymbol{u} \in \F^n$.
  
  Now we show that $\ell(\mat M_G(\alpha))$ is not invertible for all
  choices of $\ell$. By Equation \eqref{Eq:PrimElm}, we can write
  $$\mat M_G(\alpha) = \sum_{j = 1}^n \mat M^{(j)} \bar x^j, \quad 
  \mat M^{j} \in M_{n}(\F) \text{~for all $j$}.$$ Now
  $\mat M_G(\alpha) \boldsymbol{u} =0$ yields
  $\mat M^{(j)}\boldsymbol{u} = 0$ for
  $j \in \lbrace 1, \ldots , n \rbrace$. Hence,
$$\sum_{j = 1}^n \mat M^{(j)} \ell_j \boldsymbol{u} = 0$$ for any 
$\ell_j$'s in $\F$, and $\ell(\mat M_G(\alpha))$ is not invertible for
any~$\ell$.
\end{proof}

Hence, our algorithm is can be sketched as follows: choose random $\alpha$
in $\K$ and $\ell: \K\to\F$, and let
$$s_{\alpha,\ell}:=\sum_{g \in G} \ell(g(\alpha))g \in \F[G].$$ The
matrix $\ell(\mat M_G (\alpha))$ is the multiplication matrix by
$s_{\alpha,\ell}$ in $\F[G]$, so once $s_{\alpha,\ell}$ is known, we are
left with testing the whether it is a unit in $\F[G]$.

In the next two sections, we address the two questions highlighted above:
computing $s_{\alpha,\ell}$, and an invertibility test in $\F[G]$.


%%% Local Variables:
%%% mode: latex
%%% TeX-master: "NormalBasisCharZero"
%%% End:
