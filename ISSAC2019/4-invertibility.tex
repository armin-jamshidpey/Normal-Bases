\section{Testing Invertibility}
%in the Group Algebra}
\label{sec:invertibility}

In this section we consider the problem of invertibility testing in
$\F[G]$, specifically for abelian and metacyclic groups $G$: given an
element $\beta$ in $\F[G]$, for a field $\F$ and a group $G$, determine
whether $\beta$ is a unit in $\F[G]$.  
Since we are in characteristic zero, Wedderburn's theorem implies the
existence of an $\F$-algebra isomorphism (which we will refer to as a
Fourier Transform)
\[
  \F[G] \to M_{d_1}(D_1) \times \dots \times M_{d_r}(D_r),
\]
where all $D_i$'s are division algebras over $\F$. If we were working over
$\F=\C$, all $D_i$'s would simply be $\C$ itself.  A natural solution
to test the invertibility of $\beta \in \F[G]$ would then be to compute its
Fourier transform and test whether all its components
$\beta_1 \in M_{d_1}(\C),\dots,\beta_r \in M_{d_r}(\C)$ are
invertible. This boils down to linear algebra over $\C$, and takes
$O(d_1^\omega + \cdots + d_r^\omega)$ operations.  Since
$d_1^2 + \cdots + d_r^2 = |G|$, this is $O(|G|^{\omega/2})$ operations in
$\C$.

However, we do not wish to make such a strong assumption as $\F=\C$. Since
we measure the cost of our algorithms in $\F$-operations, the direct
approach that embeds $\F[G]$ into $\C[G]$ does not make it possible to
obtain a subquadratic cost in general. If, for instance, $\F=\Q$ and $G$ is
cyclic of order $n=2^k$, computing the Fourier Transform of $\beta$
requires we work in a degree $n/2$ extension of $\Q$, implying a quadratic
runtime.

We give algorithms for the problem of invertibility testing for the
families of groups seen so far, abelian and metacyclic. For the former,
we prove a stronger result: starting from a suitable presentation of
$G$, we give a softly linear-time algorithm to find an isomorphic
image of $\beta \in \F[G]$ in a product of $\F$-algebras of the form
$\F[z]/\langle P_i(z)\rangle$, for certain polynomials $P_i \in \F[z]$
(recovering $\beta$ from its image is softly-linear time as well). Not
only does this allow us to test whether $\beta$ is invertible, this
would also make it possible to find its inverse in $\F[G]$ in
softly-linear time.  For metacyclic groups, we describe an injective
$\F$-algebra homomorphism from $\F[G]$ to a matrix algebras over a
cyclotomic ring. The codomain is in general of dimension higher than
$|G|$, so the algorithm we deduce from this is not linear-time.

%%%%%%%%%%%%%%%%%%%%%%%%%%%%%%%%%%%%%%%%%%%%%%%%%%%%%%%%%%%%

\subsection{Abelian groups}

Because an abelian group is a product of cyclic groups, its group
algebra $\F[G]$ is the tensor product of cyclic algebras, so it admits
a description of the form $\F[x_1,\dots,x_t]/\langle
x_1^{n_1}-1,\dots,x_t^{n_t}-1\rangle$, for some integers
$n_1,\dots,n_t$. The complexity of arithmetic operations in an
$\F$-algebra such as $\A:=\F[x_1,\dots,x_t]/\langle
P_1(x_1),\dots,P_t(x_t)\rangle$ is difficult to pin down
precisely. For general $P_i$'s, the cost of multiplication in $\A$ is
known to be $O(\dim(\A)^{1+\varepsilon})$, for any $\varepsilon >
0$~\cite[Theorem~2]{LiMoSc09}. From this it may be possible to deduce
similar upper bounds on the complexity of invertibility tests,
following~\cite{DaMMMScXi06}, but this seems non-trivial.

Instead, we give an algorithm with softly linear runtime, that uses
the factorization properties of cyclotomic polynomials and Chinese
remaindering techniques to transform our problem into that of
invertibility testing in algebras of the form $\F[z]/\langle P_i(z)
\rangle$, for various polynomials $P_i$. The reference~\cite{Pol94}
also discusses the factors of algebras such as
$\F[x_1,\dots,x_t]/\langle x_1^{n_1}-1,\dots,x_t^{n_t}-1\rangle$, but
the resulting algorithms are different (and the cost of the
\citeauthor{Pol94}'s \citeyear{Pol94} algorithm is only known to be
polynomial in $|G|$).

\smallskip

\noindent{\bf Tensor product of two cyclotomic rings: coprime orders.}
The following proposition will be the key to foregoing multivariate
polynomials, and replacing them by univariate ones.  Let $m,m'$ be two
coprime integers and define
$$\mathbf{h}:=\F[x,x']/\langle \Phi_{m}(x), \Phi_{m'}(x')\rangle,$$
where for $i \ge 0$, $\Phi_i$ is the cyclotomic polynomial of order
$i$. In what follows, $\varphi$ is Euler's totient function, so that
$\varphi(i) = \deg(\Phi_i)$ for all~$i$.
\begin{lemma}
  There exists an $\F$-algebra isomorphism $\gamma: \mathbf{h} \to
  \F[z]/\langle\Phi_{mm'}(z)\rangle$ given by $xx' \mapsto z$.  Given
  $\Phi_m$ and $\Phi_{m'}$, $\Phi_{mm'}$ can be computed in time
  $\tilde{O}(\varphi(mm'))$; given these polynomials, one can
  apply $\gamma$ and its inverse to any input using
  $\tilde{O}(\varphi(mm'))$ operations in~$\F$.
\end{lemma}
\noindent {\sc Proof.}
  Without loss of generality, we prove the first claim over $\Q$; the
  result over $\F$ follows by scalar extension. In the field \sloppy
  $\Q[x,x']/\langle \Phi_{m}(x), \Phi_{m'}(x')\rangle$, $xx'$ is
  cancelled by $\Phi_{mm'}$. Since this polynomial is irreducible, it
  is the minimal polynomial of $xx'$, which is thus a primitive
  element for $\Q[x,x']/\langle \Phi_{m}(x),
  \Phi_{m'}(x')\rangle$. This proves the first claim.

  For the second claim, we first determine the images of $x$ and $x'$
  by $\gamma$. Start from a B\'ezout relation $am+ a'm'=1$, for some
  $a,a'$ in $\Z$.  Since $x^m = {x'}^{m'}=1$ in $\mathbf{h}$, we
  deduce that $\gamma(x)=z^{u}$ and $\gamma(x') = z^{v}$, with $u:=am
  \bmod mm'$ and $v:=a'm' \bmod mm'$. To compute $\gamma(P)$, for some
  $P$ in $\mathbf{h}$, we first compute $P(z^u, z^v)$, keeping all
  exponents reduced modulo $mm'$. This requires no arithmetic
  operations and results in a polynomial $\bar P$ of degree less than
  $mm'$, which we eventually reduce modulo $\Phi_{mm'}$ (the latter is
  obtained by the composed product algorithm of~\cite{BoFlSaSc06} in
  quasi-linear time).  By~\cite[Theorem~8.8.7]{BacSha96}, we have the
  bound $s \in O(\varphi(s) \log(\log(s)))$, so that $s$ is in
  $\tilde{O}(\varphi(s))$. Thus, we can reduce $\bar P$ modulo
  $\Phi_{mm'}$ in $\tilde{O}(\varphi(mm'))$ operations, establishing
  the cost bound for $\gamma$.

  Conversely, given $Q$ in $\F[z]/\langle\Phi_{mm'}(z)\rangle$, we obtain
  its preimage by replacing powers of $z$ by powers of $xx'$, reducing all
  exponents in $x$ modulo $m$, and all exponents in $x'$ modulo $m'$.  We
  then reduce the result modulo both $\Phi_m(x)$ and $\Phi_{m'}(x')$.  By
  the same argument as above, the cost is softly linear in $\varphi(mm')$.
\qed

\noindent{\bf Extension to several cyclotomic rings.}  The natural
generalization of the algorithm above starts with pairwise distinct
primes $\boldsymbol{p}=(p_1,\dots,p_t)$, non-negative exponent
$\boldsymbol{c}=(c_1,\dots,c_t)$ and variables
$\boldsymbol{x}=(x_1,\dots,x_t)$ over $\F$. Now, we define
$$\H:=\F[x_1,\dots,x_t]/\langle
\Phi_{{p_1}^{c_1}}(x_1),\dots,\Phi_{{p_t}^{c_t}}(x_t)\rangle;$$ when
needed, we will write $\H$ as
$\H_{\boldsymbol{p},\boldsymbol{c},\boldsymbol{x}}$. Finally, we let
$\mu:={p_1}^{c_1}\cdots {p_t}^{c_t}$; then, the dimension $\dim(\H)$ is
$\varphi(\mu)$.

\begin{lemma}\label{lemma:distinctP}
 There exists an $\F$-algebra isomorphism $\Gamma: \H \to
 \F[z]/\langle\Phi_{\mu}(z)\rangle$ given by $x_1 \cdots x_t \mapsto
 z$.  One can apply $\Gamma$ and its inverse to any input using
 $\tilde{O}(\dim(\H))$ operations in $\F$.
\end{lemma}
\noindent {\sc Proof.}
  We proceed iteratively. First, note that the cyclotomic polynomials
  $\Phi_{{p_i}^{c_i}}$ can all be computed in time $O(\varphi(\mu))$. 
  The isomorphism
  $\gamma: \F[x_1,x_2]/\langle \Phi_{{p_1}^{c_1}}(x_1),
  \Phi_{{p_2}^{c_2}}(x_2)\rangle \to \F[z]/\langle
  \Phi_{{p_1}^{c_1}{p_2}^{c_2}}(z)\rangle$
given in the previous paragraph extends coordinate-wise to an
  isomorphism
  $$\Gamma_1: \H \to \F[z,x_3,\dots,x_t]/\langle
  \Phi_{{p_1}^{c_1}{p_2}^{c_2}}(z),\Phi_{{p_3}^{c_3}}(x_3),\dots,\Phi_{{p_t}^{c_t}}(x_t)\rangle.$$
  By the previous lemma, $\Gamma_1$ and its inverse can be applied to
  any input in time $\tilde{O}(\varphi(\mu))$. Iterate this process
  another $t-2$ times, to obtain $\Gamma$ as a product
  $\Gamma_{t-1} \circ \cdots \circ \Gamma_1$. Since $t$ is logarithmic 
  in $\varphi(\mu)$, the proof is complete.
\qed

\noindent{\bf Tensor product of two prime-power cyclotomic rings, same
  $p$.}
We now consider cyclotomic polynomials of prime power
orders for a common prime $p$. As above, we start with two such polynomials.
Let thus $p$ be a prime. The key to the following algorithms is the
lemma below.  Let $c,c'$ be positive integers, with $c \ge
c'$, and let $x,y$ be indeterminates over $\F$. Define
\begin{align}
\mathbf{I}&:=\F[x]/\Phi_{p^c}(x),  \\
\mathbf{J}&:=\F[x,y]/\langle \Phi_{p^c}(x), \Phi_{p^{c'}}(y)\rangle = \mathbf{I}[y]/\Phi_{p^{c'}}(y).
\end{align}
Note $\mathbf{I}$ and $\mathbf{J}$ have respective dimensions
$\varphi(p^c)$ and $\varphi(p^c) \varphi(p^{c'})$.

\smallskipback
\begin{lemma}
  There is an $\F$-algebra isomorphism $\theta: \mathbf{J} \to
  \mathbf{I}^{\varphi(p^{c'})}$ such that one can apply $\theta$ or
  its inverse to any inputs using $\tilde{O}(\dim(\mathbf{J}))$ operations in $\F$.
\end{lemma}

\smallskipback
\noindent{\sc Proof.}
  Let $\xbar$ be the residue class of
  $x$ in $\A$. Then, in $\mathbf{I}[y]$, $\Phi_{p^{c'}}(y)$ factors as
  $$\Phi_{p^{c'}}(y) =\prod_{\substack{1 \le i\le p^{c'}-1\\ \gcd(i,p)
      =1}} (y-\rho_i),$$ with $\rho_i:={\xbar}^{i p^{c-c'}}$ for all
  $i$.  Even though $\mathbf{I}$ may not be a field, the Chinese
  Remainder theorem implies that $\mathbf{J}$ is isomorphic to
  $\mathbf{I}^{\varphi(p^{c'})}$; the isomorphism is given by
  $$\begin{array}{cccc}
    \theta: & \mathbf{J} & \to & \mathbf{I} \times \cdots \times \mathbf{I}, \\
    & P & \mapsto& (P(\xbar,\rho_1),\dots,P(\xbar,\rho_{\varphi(p^{c'})}).
  \end{array}$$
  Arithmetic operations $(+,-,\times)$ in
  $\mathbf{I}$ can be done in $\tilde{O}(\varphi(p^c))$ operations
  in $\F$. Starting from $\rho_1 \in \mathbf{I}$, all other roots
  $\rho_i$ can then be computed in $O(\varphi(p^{c'}))$ operations in
  $\mathbf{I}$ or $\tilde{O}(\dim(\mathbf{J}))$
  operations in~$\F$. 
  
Applying $\theta$ and its inverse is done by means of fast evaluation
and interpolation~\cite[Chapter~10]{vzGathen13} in $\tilde{O}(\varphi(p^{c'}))$
operations in $\mathbf{I}$, that is, $\tilde{O}(\deg(\mathbf{J}))$ operations in $\F$
(the algorithms do not require that $\mathbf{I}$ be a field).
\qed

\smallskip\noindent{\bf Extension to several cyclotomic rings.}
Let $p$ be as before, and consider now non-negative integers
$\boldsymbol{c}=(c_1,\dots,c_t)$ and variables $\boldsymbol{x}=(x_1,\dots,x_t)$. We
define the $\F$-algebra
$$\A:=\F[x_1,\dots,x_t]/\langle \Phi_{p^{c_1}}(x_1), \dots,
\Phi_{p^{c_t}}(x_t)\rangle,$$ which we will sometimes write
$\A_{p,\boldsymbol{c},\boldsymbol{x}}$ to make the dependency on $p$
and the $c_i$'s clear. Up to reordering the $c_i$'s, we can assume
that $c_1 \ge c_i$ holds for all $i$, and define as before
$\mathbf{I}:=\F[x_1]/\Phi_{p^{c_1}}(x_1)$.

\begin{lemma}\label{lemma:A}
  There exists an $\F$-algebra isomorphism $\Theta: \A \to
  \mathbf{I}^{\dim(\A)/\dim(\mathbf{I})}$. This isomorphism and its
  inverse can be applied to any inputs using $\tilde{O}(\dim(\A))$
  operations in $\F$.
\end{lemma}
\noindent{\sc Proof.}
Without loss of generality, we can assume that all $c_i$'s are non-zero
(since for $c_i=0$, $\Phi_{p^{c_i}}(x_i)=x_i-1$,
so $\F[x_i]/\langle \Phi_{p^{c_i}}(x_i) \rangle = \F$).
We proceed iteratively. First, rewrite $\A$ as
$$\A=\mathbf{I}[x_2,x_3,\dots,x_t]/\langle \Phi_{p^{c_2}}(x_2), \Phi_{p^{c_3}}(x_3), \dots,
\Phi_{{p_t}^{c_t}}(x_t)\rangle.$$ 
The isomorphism 
$\theta: \mathbf{I}[x_2]/\Phi_{p^{c_2}}(x_2) \to \mathbf{I}^{\varphi(p^{c_2})}$
introduced in the previous paragraph extends coordinate-wise
to an isomorphism 
$$\Theta_1: \A \to (\mathbf{I}[x_3,\dots,x_t]/\langle
\Phi_{p^{c_3}}(x_3), \dots,
\Phi_{p^{c_t}}(x_t)\rangle)^{\varphi(p^{c_2})};$$ $\Theta_1$ and its
inverse can be evaluated in quasi-linear time $\tilde{O}(\dim(\A))$.
We now work in all copies of $\mathbf{I}[x_3,\dots,x_t]/\langle
\Phi_{p^{c_3}}(x_3), \dots, \Phi_{p^{c_t}}(x_t)\rangle$ independently,
and apply the procedure above to each of them. Altogether we have
$t-1$ such steps to perform, giving us an isomorphism
$$\Theta = \Theta_{t-1} \circ \cdots \circ \Theta_1:
\A \to
\mathbf{I}^{\varphi(p^{c_2}) \cdots \varphi(p^{c_t})}.$$
The exponent can be rewritten as $ \dim(\A)/\dim(\mathbf{I})$, as claimed.
All $\Theta_i$'s and their inverses can be computed
in time $\tilde{O}(\dim(\A))$, and we do $t-1$ of them,
where $t$ is $O(\log(\dim(\A)))$. 
\qed

\noindent{\bf Decomposing certain $p$-group algebras.}  The prime $p$
and indeterminates $\boldsymbol{x}=(x_1,\dots,x_t)$ are as before; we now consider
positive integers $\boldsymbol{b}=(b_1,\dots,b_t)$, and the $\F$-algebra
\[
\begin{array}{ccl}
\B&:=&\F[x_1,\dots,x_t]/\langle x_1^{p^{b_1}}-1,\dots,x_t^{p^{b_t}}-1\rangle\\$$
&=& \F[x_1]/\langle x_1^{p^{b_1}}-1 \rangle \otimes \cdots \otimes \F[x_t]/\langle x_t^{p^{b_t}}-1 \rangle.
\end{array}
\]
If needed, we will write $\B_{p,\boldsymbol{b},\boldsymbol{x}}$ to make the dependency
on $p$ and the $b_i$'s clear. This is the $\F$-group algebra
of $\Z/p^{b_1}\Z \times \cdots \times \Z/p^{b_t}\Z$.

\begin{lemma}\label{lemma:alg}
  There exists a positive integer $N$, non-negative integers
  $\boldsymbol{c}=(c_1,\dots,c_N)$ and  an
  $\F$-algebra isomorphism 
  $$\Lambda: \B \to \D= \F[z]/\langle \Phi_{p^{c_1}}(z) \rangle \times \cdots \times \F[z]/\langle \Phi_{p^{c_N}}(z)\rangle.$$
  One can apply the isomorphism and its inverse to any 
  input using $\tilde{O}(\dim(\B))$ operations in $\F$.
\end{lemma}
\noindent{\sc Proof.}
For $i \le t$, we have the factorization
$$x_i^{p^{b_i}}-1 = \Phi_1(x_i) \Phi_p(x_i) \Phi_{p^2}(x_i) \cdots
\Phi_{p^{b_i}}(x_i);$$ note that $\Phi_1(x_i)=x_i-1$.  The factors may
not be irreducible, but are pairwise coprime, so we have a
Chinese Remainder isomorphism
\[
  \lambda_i: \F[x_i]/\langle x_i^{p^{b_i}}-1 \rangle \to \F[x_i]/\langle \Phi_1(x_i)\rangle
  \times \cdots \times  \F[x_i]/\langle \Phi_{p^{b_i}}(x_i)\rangle.
\]
It and its inverse can be computed  
in $\tilde{O}(p^{b_i})$ operations in $\F$~\cite[Chapter~10]{vzGathen13}. 
This gives an $\F$-algebra isomorphism
$$\lambda: \B \to \prod_{c_1=0}^{b_1} \cdots \prod_{c_t=0}^{b_t} \A_{p,\boldsymbol{c},\boldsymbol{x}},$$
with $\boldsymbol{c}=(c_1,\dots,c_t)$. Together with its inverse, 
$\lambda$ can be computed in $\tilde{O}(\dim(\B))$ operations in $\F$.
Composing with the result in Lemma~\ref{lemma:A}, this gives
us an isomorphism
$$\Lambda: \B \to \D:=\prod_{c_1=0}^{b_1} \cdots \prod_{c_t=0}^{b_t}
\mathbf{I}_{\boldsymbol{c}}^{D_{\boldsymbol{c}}},$$ where
$\mathbf{I}_{\boldsymbol{c}} = \F[z]/\langle \Phi_{p^c}(z)\rangle$,
with $c =\max(c_1,\dots,c_t)$ and $D_{\boldsymbol{c}} =
\dim(\A_{t,\boldsymbol{c},\boldsymbol{x}})/\dim(\mathbf{I}_{\boldsymbol{c}})$. As
before, $\Lambda$ and its inverse can be computed in quasi-linear time
$\tilde{O}(\dim(\B))$. \qed

As for $\B$, we will write $\D_{p,\boldsymbol{b},\boldsymbol{x}}$ if needed; it is
well-defined, up to the order of the factors.

\smallskip

\noindent{\bf Main result.} Let $G$ be an abelian group.  We can
write the elementary divisor decomposition of $G$ as $G = G_1 \times
\cdots \times G_s$, where each $G_i$ is of prime power order
$p_i^{a_i}$, for pairwise distinct primes $p_1,\dots,p_s$, so that
$|G| = p_1^{a_1} \cdots p_s^{a_s}$. Each $G_i$ can itself be written
as a product of cyclic groups, $G_i = G_{i,1} \times \cdots \times
G_{i,t_i}$, where the factor $G_{i,j}$ is cyclic of order
${p_i}^{b_{i,j}}$, with $b_{i,1} \le \cdots \le b_{i,t_i}$ and $b_{i,1} + \cdots +
b_{i,t_i} = a_i$. We henceforth assume that generators
$\gamma_{1,1},\dots,\gamma_{s,t_s}$ of respectively
$G_{1,1},\dots,G_{s,t_s}$ are known, and that elements of $\F[G]$ are
given on the power basis in $\gamma_{1,1},\dots,\gamma_{s,t_s}$. 

%% Were
%% this not the case, given arbitrary generators $g_1,\dots,g_r$ of $G$, with
%% orders $e_1,\dots,e_r$, a brute-force solution would factor each $e_i$
%% (factoring $e_i$ takes $o(e_i)$ bit operations on a standard RAM), so
%% as to write $\langle g_i \rangle$ as a product of cyclic groups of
%% prime power orders, from which the required decomposition follows.

\begin{proposition}
  Given $\beta \in \F[G]$, written in the basis
  $\gamma_{1,1},\dots,\gamma_{s,t_s}$, one can test if $\beta$ is a
  unit in $\F[G]$ in time $\tilde{O}(|G|)$.
\end{proposition}
From the factorization $G = G_1 \times \cdots \times G_s$, we deduce
that the group algebra $\F[G]$ is the tensor product $\F[G_1]
\otimes \cdots \otimes \F[G_s]$. Furthermore, the 
factorization $G_i = G_{i,1} \times \cdots \times G_{i,t_i}$
implies that $\F[G_i]$ is isomorphic, as an $\F$-algebra, to
$$\F[x_{i,1},\dots,x_{i,t_i}]/\left \langle
x_{i,1}^{p_i^{b_{1}}}-1,\dots,x_{i,t_i}^{p_i^{b_{i,t_i}}}-1\right\rangle
=\B_{p_i,\boldsymbol{b}_i,\boldsymbol{x}_i},$$ with $\boldsymbol{b}_i
= (b_{i,1},\dots,b_{i,t_i})$ and $\boldsymbol{x}_i =
(x_{i,1},\dots,x_{i,t_i})$. Given $\beta$ on the power basis in
$\gamma_{1,1},\dots,\gamma_{s,t_s}$, we obtain its image $B$ in
$\B_{p_1,\boldsymbol{b}_1,\boldsymbol{x}_1} \otimes \cdots \otimes
\B_{p_s,\boldsymbol{b}_s,\boldsymbol{x}_s}$ simply by renaming
$\gamma_{i,j}$ as $x_{i,j}$, for all $i,j$.

For $i \le s$, by Lemma~\ref{lemma:alg}, there exist integers
$c_{i,1},\dots,c_{i,N_i}$ such that
$\B_{p_i,\boldsymbol{b}_i,\boldsymbol{x}_i}$ is isomorphic to an
algebra $\D_{p_i, \boldsymbol{b}_i, z_i}$, with factors 
$\F[z_i]/\langle \Phi_{{p_i}^{c_{i,j}}}(z_i) \rangle$.
By distributivity of the tensor product over direct products, we
deduce that $\B_{p_1,\boldsymbol{b}_1,\boldsymbol{x}_1} \otimes \cdots
\otimes \B_{p_s,\boldsymbol{b}_s,\boldsymbol{x}_s}$ is isomorphic to
the product of algebras
 \begin{equation}\label{eq:prod}
\text{\small $\prod$}_{\boldsymbol{j}}~ \F[z_1,\dots,z_s]/
\langle \Phi_{{p_1}^{c_{1,j_1}}}(z_1),\dots, \Phi_{{p_s}^{c_{s,j_s}}}(z_s) \rangle,   
 \end{equation}
for all indices $\boldsymbol{j}=(j_1,\dots,j_s)$, with
$j_1 =1,\dots,N_1,\dots,j_s=1,\dots,N_s$;
call $\Gamma$ the isomorphism. Given $B$ in $\B_{p_1,\boldsymbol{b}_1,\boldsymbol{x}_1} \otimes
\cdots \otimes \B_{p_s,\boldsymbol{b}_s,\boldsymbol{x}_s}$,
Lemma~\ref{lemma:alg} also implies that $B':=\Gamma(B)$ can be
computed in time $\tilde{O}(|G|)$ (apply the isomorphism
corresponding to $\boldsymbol{x}_1$ coordinate-wise with respect to
all other variables, then deal with $\boldsymbol{x}_2$, etc).
The codomain in~\eqref{eq:prod} is the product of all $\H_{\boldsymbol{p},\boldsymbol{c}_{\boldsymbol{j}},\boldsymbol{z}}$,
with 
$$\boldsymbol{p}=(p_1,\dots,p_s),\quad \boldsymbol{c}=(c_{1,j_1},\dots,c_{s,j_s}),\quad \boldsymbol{z}=(z_1,\dots,z_s).$$
Apply Lemma~\ref{lemma:distinctP} to all 
$\H_{\boldsymbol{p},\boldsymbol{c}_{\boldsymbol{j}},\boldsymbol{z}}$ to obtain
an $\F$-algebra isomorphism
$$\Gamma': \text{\small $\prod$}_{\boldsymbol{j}}~
\H_{\boldsymbol{p},\boldsymbol{c}_{\boldsymbol{j}},\boldsymbol{z}} \to
\text{\small $\prod$}_{\boldsymbol{j}} ~\F[z]/\langle
\Phi_{d_{\boldsymbol{j}}}(z) \rangle,$$ for certain integers
$d_{\boldsymbol{j}}$. The lemma implies that given $B'$,
$B'':=\Gamma'(B')$ can be computed in softly linear time
$\tilde{O}(|G|)$ as well. Invertibility of $\beta \in \F[G]$ is
equivalent to $A''$ being invertible, that is, to all its components
being invertible in the respective factors $\F[z]/\langle
\Phi_{d_{\boldsymbol{j}}}(z) \rangle$. Invertibility in such an
algebra can be tested in softly linear time by applying the fast
extended GCD algorithm~\cite[Chapter~11]{vzGathen13}, so our conclusion follows. 
With Proposition \ref{prop:abelian}, this proves
the first part of Theorem \ref{thm:main}.

%%%%%%%%%%%%%%%%%%%%%%%%%%%%%%%%%%%%%%%%%%%%%%%%%%%%%%%%%%%%

\subsection{Metacyclic Groups}

We next study the invertibility problem for a
metacyclic group $G$. We use an injective homomorphism, whose image will
be easy to compute. This is the object of the following lemma, where
the map is inspired by the one used in \cite[\S 47]{Curtis}.

Assume that $G = \langle \sigma , \tau : \sigma^m = 1, \tau^s =
\sigma^t, \tau^{-1} \sigma \tau = \sigma^r \rangle$, where $r^s = 1
\bmod m$ and $rt = t \bmod m$; in particular, $n=|G|$ is equal to
$ms$. Define $\A:=\F[z]/\langle z^m-1\rangle$ and let $\zbar$ be the
image of $z$ in $\A$.

\begin{lemma}\label{prop:metinjection}
The mapping $\psi: \F [G] \to M_s(\A)$ where
\[
\sigma \mapsto \mathrm{Diag}(\zbar, \zbar^r , \ldots, \zbar^{r^{s-1}})
,\; \tau \mapsto 
\left[ \begin{array}{l|l}
0 & \zbar\\
\hline
\mat I_{s-1} & 0
\end{array}
\right]
\]
is an injective $\F$-algebra homomorphism.
\end{lemma}
\noindent{\sc Proof.}
It is straightforward to verify that $\psi(\sigma)^m = \mat I_m$,
$\psi(\tau)^s = \psi(\sigma)^t$ and $\psi_i(\sigma) \psi_i(\tau)
=\psi(\tau) \psi_i(\sigma)^r$; this shows that $\psi$ is a well-defined $\F$-algebras homomorphism.

Take $\beta \in \F[G]$, and write it $\beta = \sum_{j = 0}^{s-1}
\left( \sum_{i = 0}^{m-1} b_{i,j} \sigma^i \right) \tau^j$. For
$j=0,\dots,s-1$, define $F_j(x) := \sum_{i = 0}^{m-1} b_{i,j}x^i \in
\F[x]$ and, for $1 \leq i,j \leq s$,
$F_{i,j} := F_{i-1}(\zbar^{r^{j-1}}).$
Then, $\psi(\beta)$ is the matrix
\begin{equation}\label{eq:injection}
\left[\begin{array}{llll}
F_{1,1} &  \cdots	&	\zbar F_{3,s-1} & \zbar F_{2,1}\\
F_{2,2} & F_{1,2}& \cdots & \zbar F_{3,s}\\
\vdots &\ddots & \ddots& \vdots\\
F_{s,s}& \cdots & F_{2,s}	& F_{1,s}
\end{array}
\right].
\end{equation}
If $\beta$ is in $\mathrm{Ker}(\psi)$, we get $F_i(\zbar) =0$, that is,
$F_i \bmod (z^m-1)=0$, for $0 \leq i < s$.  All $F_i$'s have degree 
less than $m$, so they are all zero. \qed


We finally give two algorithms that test whether $\psi(\beta) \in
M_s(\A)$ is invertible, for a given $\beta$ in $\F[G]$. Minor
difficulties will arise as we work over $\A$, since $\A$ is not a
field, but a product of fields (if the irreducible factorization of
$z^m-1$ in $\F[z]$ is known, we can use the Chinese Remainder theorem
and work in field extensions of $\F$).

\begin{corollary}\label{coro:test_meta}
  Given $\beta$ in $\F[G]$, one can test if $\beta$ is a unit in
  $\F[G]$ either by a deterministic algorithm that uses
  $\tilde{O}(s^{2.7} m)$ operations in $\F$, or a Monte Carlo one that
  uses $\tilde{O}(n^2)$ operations in $\F$.
\end{corollary}
The second statement provides the last part of the proof of Theorem
\ref{thm:main}. Note that the first algorithm gives a better cost in
many cases. For instance, if $s \leq m$, the first algorithm uses
$O(n^{1.85})$ operations in $\F$. This happens if $s$ is prime, since
then the number ${(m- \gcd(m,r-1))}/{s}$ is a positive integer,
which implies $s \leq m$ (see \cite[Theorem 47.12, Corollary 47.14
]{Curtis}).


\noindent{\sc First algorithm.}
  The first algorithm uses fast linear algebra algorithms over the ring
  $\A$. Here, we start from $\beta$ written as $\beta = \sum_{j =
    0}^{s-1} \left( \sum_{i = 0}^{m-1} b_{i,j} \sigma^i \right) \tau^j
  \in \F[G]$. Then, the proof of the previous lemma shows an explicit
  formula for $\psi(\beta)$. In order to compute this matrix, we note
  that $\zbar^{r^{j-1}} = \zbar^{r^{j-1} \bmod m}$; computing this
  element and its powers requires no arithmetic operation, so that the
  coefficients of each $F_{i,j}$ are obtained in linear time $O(m)$.
  Hence the matrix $\psi(\beta)$ can be computed in time $O(s^2m)$.

  Next, we have to determine whether $\psi(\beta)$ is a unit (the
  injectivity of $\psi$ implies that this is the case if and only if
  $\beta$ itself is a unit). This amounts to computing the determinant
  of this matrix, which can be done in $\tilde{O}(s^{2.7} m)$
  operations in $\F$, using the determinant algorithm
  of~\cite[Section~6]{KaVi04}. \qed


%% Before giving our second algorithm, let us point out that matrix-vector
%% products by $\psi(\beta)$ can be done fast.

\begin{lemma}\label{lem:multpsi}
  Given $\beta$ in $\F[G]$ and $\boldsymbol{v}$ in $\A^s$, one can
  compute $\psi(\beta) \boldsymbol{v} \in \A^s$ using $\tilde{O}(s
  m^2)$ operations in $\F$.
\end{lemma}

\smallskipback
\noindent{\sc Proof.}
  We use the basis of  $\F[G]$ of~\eqref{pres2}, writing $\beta =
  \sum_{i = 0}^{m-1} \left( \sum_{j = 0}^{s-1}  b_{i,j} \tau^i
  \right) \sigma^j = \sum_{i
    = 0}^{m-1} B_i(\tau) \sigma^i$, for some $B_0,\dots,B_{m-1}$ in
  $\F[z]$ of degree less than $s$.

 Given $\boldsymbol{v}$ as above, we compute all $B_i(\psi(\tau))
 \psi(\sigma)^i \boldsymbol{v}$ independently, and add them to obtain
 $\psi(\beta) \boldsymbol{v}$. Hence, let us fix an index $i$ in
 $\{0,\dots,m-1\}$.
 The vector $\psi(\sigma)^i \boldsymbol{v}$ can be obtained by
 multiplying each entry of $\boldsymbol{v}$ by a power of $\zbar$;
 this takes $\tilde{O}(sm)$ operations in $\F$. Then, since
 $\psi(\tau)$ is the matrix of multiplication by $y$ in $\A[y]/\langle
 y^s-\zbar\rangle$, $B_i(\psi(\tau))$ is the matrix of multiplication
 by $B_i(y)$ in $\A[y]/\langle y^s-\zbar\rangle$. Thus, applying this
 matrix to a vector also takes time $\tilde{O}(sm)$.
 Adding a factor of $m$ to account for all indices $i$ gives
 the result.
\qed





\noindent{\sc Second algorithm for Corollary~\ref{coro:test_meta}.}
  The second algorithm uses \citeauthor{Wiedemann86}'s
  \citeyear{Wiedemann86} algorithm,
  and its extension by~\citeN{KaSa91}. Extra care will be needed to
  accommodate the fact that $\A$ has zero-divisors. Let
  $F_1,\dots,F_s$ be the (unknown) irreducible factors of $z^m-1$ in
  $\F[z]$ and define $\A_i := \F[z]/\langle F_i \rangle$ for
  $i=1,\dots,s$. We write $\pi_i: \A \to\A_i$ for the canonical
  projection, and extend the notation to matrices over $\A$.

  For $\beta$ in $\F[G]$, $\mat M:=\psi(\beta)$ is invertible if and only
  if all $\mat M_i := \pi_i(\mat M)$ are. We are going to use the algorithm
  of~\cite[Section~4]{KaSa91} to compute the rank of all these matrices
  (these ranks are well-defined, since all $\A_i$'s are fields).  Let
  $\mat L$ and $\mat U$ be respectively random lower triangular and upper
  triangular Toeplitz matrices over $\A$, and define
  $\mat M':=\mat L \mat M \mat U \in M_s(\A)$. Finally, let $\mat M''$ be
  $\mat M'$, to which we adjoin a bottom row and a rightmost column of
  zeros (so it has size $s+1$), let $\mat M''_i :=\pi_i(\mat M'')$ and let
  $r_i :=\text{rank}(\mat M''_i)$, $i=1,\dots,s$. Then, all $r_i$'s are
  less than $s+1$, and $\mat M$ is invertible if and only if $r_i=s$ for
  all $i$.

  The condition that $\mat M''_i$ has rank less than $s+1$ makes it
  possible to apply~\cite[Lemma~2]{KaSa91}: for generic
  $\boldsymbol{u}_i, \boldsymbol{v}_i$ in $\A_i^{s+1}$ and diagonal
  matrix $\mat X$ in $M_{s+1}(\A_i)$, the minimal polynomial of the
  sequence $(\boldsymbol{u}_i^T (\mat M''_i \mat X_i)^j
  \boldsymbol{v}_i)_{j \ge 0}$ has degree $r_i+1$.

  To compute these degrees without knowing the factorization $z^m-1 =
  F_1 \cdots F_s$, we choose random $\boldsymbol{u}, \boldsymbol{v}$
  in $\A^{s+1}$ and diagonal matrix $\mat X$ in $M_{s+1}(\A)$.  Then,
  we compute $2s$ terms in the sequence $(\gamma_j)_{j\ge 0}$, with
  $\gamma_j:=\boldsymbol{u}^T (\mat M'' \mat X)^j
  \boldsymbol{v}$. Since multiplication by $\mat L$, $\mat U$ and
  $\mat X$ all take quasi-linear time $\tilde{O}(sm)$,
  Lemma~\ref{lem:multpsi} shows that one product by $\mat M'' \mat X$
  takes $\tilde{O}(sm^2)$ operations in $\F$. Hence, all required
  terms can be obtained in $\tilde{O}(s^2m^2)=\tilde{O}(n^2)$
  operations in $\F$.

  Finally, we apply the fast Euclidean algorithm to $\sum_{j=0}^{2s-1}
  \gamma_j y^j$ and $y^{2s}$ in the ring $\A[y]$ to find the ranks
  $r_1,\dots,r_s$.  Since $\A$ is not a field, we rely on the
  algorithm of~\cite{AcCoMa03,DaMMMScXi06}. Using $\tilde{O}(sm)$
  operations in $\F$, it reveals a partial factorization of $z^m-1$ as
  $G_1 \cdots G_t$ (the factors may not be irreducible) and integers
  $\rho_j$, $j=1,\dots,t$, such that for all $i \le s$, $j\le t$, if
  $F_i$ divides $G_j$, then $r_i=\rho_j$. This allows us to determine all
  $r_i$'s, and thus decide whether $\psi(\beta)$ is singular.\qed






%%% Local Variables:
%%% mode: latex
%%% TeX-master: "NormalBasisCharZero"
%%% End:
