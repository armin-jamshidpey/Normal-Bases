\documentclass[sigconf]{acmart}

\usepackage{booktabs} % For formal tables
\usepackage{tikz}
\usepackage{mathdots}
\usepackage{esvect}
\usepackage{bbm}

\usepackage{xcolor}
\definecolor{darkgreen}{rgb}{0,.35,0}
\definecolor{darkblue}{rgb}{0,0,.5}
\definecolor{darkred}{rgb}{.6,0,0}

\usepackage{hyperref}
\hypersetup{pdfauthor={Mark Giesbrecht, Armin Jamshidpey, \'Eric Schost},
  pdftitle={Quadratic Probabilistic Algorithms for Normal Bases},
  bookmarksnumbered=true,colorlinks,linkcolor=darkblue,
 citecolor=darkgreen, urlcolor=darkred}
% ]{hyperref}

\numberwithin{equation}{section}

\citestyle{acmauthoryear}

\newcommand{\osum}[2]{\alpha_{#1,#2}}
\newcommand{\osumcost}{O(n^{(3/4)\cdot \omega(4/3)})}
\newcommand{\osumcosttilde}{\tilde{O}(n^{(3/4)\cdot \omega(4/3)})}
\newcommand{\thecost}{\tilde{O}(\vert G \vert ^{(3/4)\cdot \omega(4/3)})}

\newcommand{\FF}{{\mathbb{F}}}
\newcommand{\xbar}{\xi}
\newcommand{\zbar}{\zeta}
\newcommand{\alg}{quadratic\,}

{
      \theoremstyle{acmplain}
      \newtheorem{assumption}{Assumption}
  }

{
      \theoremstyle{acmplain}
      \newtheorem{remark}{Remark}
  }


% Copyright
%\setcopyright{none}
%\setcopyright{acmcopyright}
%\setcopyright{acmlicensed}
\setcopyright{rightsretained}
%\setcopyright{usgov}
%\setcopyright{usgovmixed}
%\setcopyright{cagov}
%\setcopyright{cagovmixed}


% DOI
\acmDOI{10.475/123_4}

% ISBN
\acmISBN{123-4567-24-567/08/06}

%Conference
\acmConference[ISSAC]{International Symposium on Symbolic and Algebraic Computation}{2019}{Beijing}
\acmYear{2019}
\copyrightyear{2019}


\acmArticle{4}
\acmPrice{15.00}

% These commands are optional
%\acmBooktitle{Transactions of the ACM Woodstock conference}
%\editor{Jennifer B. Sartor}
%\editor{Theo D'Hondt}
%\editor{Wolfgang De Meuter}


\begin{document}
\title{Quadratic Probabilistic Algorithms for Normal Bases}
%\titlenote{Produces the permission block, and
%  copyright information}
%\subtitle{Extended Abstract}
%\subtitlenote{The full version of the author's guide is available as
%  \texttt{acmart.pdf} document}


\author{Mark Giesbrecht
}
\affiliation{%
  \institution{Cheriton School of Computer Science
University of Waterloo}
}
\email{mwg@uwaterloo.ca}

\author{Armin Jamshidpey}
\affiliation{%
  \institution{Cheriton School of Computer Science
University of Waterloo}
}
\email{armin.jamshidpey@uwaterloo.ca}

\author{\'Eric Schost}
\affiliation{%
  \institution{Cheriton School of Computer Science
University of Waterloo}
}
\email{eschost@uwaterloo.ca}


% The default list of authors is too long for headers.
\renewcommand{\shortauthors}{Giesbrecht, Jamshidpey, Schost}

\newcommand{\F}{{\mathsf{F}}}
\newcommand{\K}{{\mathsf{K}}}

\newcommand{\NN}{{\mathbb{N}}}
\newcommand{\N}{{\mathbb{N}}}

\def\A{\mathbb{A}}
\def\H{\mathbb{H}}
\def\B{\mathbb{B}}
\def\Z{\mathbb{Z}}
\def\C{\mathbb{C}}
\def\Q{\mathbb{Q}}
\def\D{\mathbb{D}}
\newcommand{\QQ}{\mathbb{Q}}
\newcommand{\mat}[1]{\mathbf{\MakeUppercase{#1}}} % for a matrix

\begin{abstract}
  It is well known that for any finite Galois extension field $\K/\F$,
  with Galois group $G = \mathrm{Gal}(\K/\F)$, there exists an element
  $\alpha \in \K$ whose orbit $G\cdot\alpha$ forms an $\F$-basis of
  $\K$. Such an element $\alpha$ is called \emph{normal} and
  $G\cdot\alpha$ is called a normal basis. In this paper we introduce
  a probabilistic algorithm for finding a normal element when $G$ is
  either a finite abelian or a metacyclic group. The algorithm is
  based on the fact that deciding whether a random element $\alpha \in
  \K$ is normal can be reduced to deciding whether $\sum_{\sigma \in
    G} \sigma(\alpha)\sigma \in \K[G]$ is invertible. In an algebraic
  model, the cost of our algorithm is quadratic in the size of $G$ for
  metacyclic $G$ and slightly subquadratic for abelian $G$.
\end{abstract}

%
% The code below should be generated by the tool at
% http://dl.acm.org/ccs.cfm
% Please copy and paste the code instead of the example below. 
%
\begin{CCSXML}
<ccs2012>
 <concept>
  <concept_id>10010520.10010553.10010562</concept_id>
  <concept_desc>Computer systems organization~Embedded systems</concept_desc>
  <concept_significance>500</concept_significance>
 </concept>
 <concept>
  <concept_id>10010520.10010575.10010755</concept_id>
  <concept_desc>Computer systems organization~Redundancy</concept_desc>
  <concept_significance>300</concept_significance>
 </concept>
 <concept>
  <concept_id>10010520.10010553.10010554</concept_id>
  <concept_desc>Computer systems organization~Robotics</concept_desc>
  <concept_significance>100</concept_significance>
 </concept>
 <concept>
  <concept_id>10003033.10003083.10003095</concept_id>
  <concept_desc>Networks~Network reliability</concept_desc>
  <concept_significance>100</concept_significance>
 </concept>
</ccs2012>  
\end{CCSXML}

%\ccsdesc[500]{Computer systems organization~Embedded systems}
%\ccsdesc[300]{Computer systems organization~Redundancy}
%\ccsdesc{Computer systems organization~Robotics}
%\ccsdesc[100]{Networks~Network reliability}
%
%
%\keywords{ACM proceedings, \LaTeX, text tagging}


\maketitle

\section{Introduction}

For a finite Galois field extension $\K/\F$, with Galois group $G =
\mathrm{Gal}(\K/\F)$, an element $\alpha \in \K$ is called
\emph{normal} if the set of its Galois conjugates $G \cdot \alpha = \{
g(\alpha): g\in G\}$ forms a basis for $\K$ as a vector space over
$\F$. The existence of normal element for any finite Galois extension
is classical, and constructive proofs are provided in most algebra
texts (see, e.g., \citep[Section 6.13]{Lang}).
%\cite{Wae70}, Section~8.11).
 
While there is a wide range of well-known applications of normal bases in
finite fields, such as fast exponentiation~\citep{GaGaPaSh00}, there also
exist applications of normal elements in characteristic zero.  For instance,
in multiplicative invariant theory, for a given permutation lattice and
related Galois extension, a normal basis is useful in computing the
multiplicative invariants explicitly~\citep{Jam18}.

A number of algorithms are available for finding a normal element in
characteristic zero and in finite fields.  Because of their immediate
applications in finite fields, algorithms for determining normal
elements in this case are most commonly seen.  A fast randomized
algorithm for determining a normal element in a finite field
$\FF_{q^n}/\FF_q$, where $\FF_{q^n}$ is the finite field with $q^n$
elements for any prime power $q$ and integer $n>1$, is presented by
\citeN{GatGie90}, with a cost of $O(n^2+n\log q)$ operations in
$\FF_q$.  A faster randomized algorithm is introduced by
\citeN{KalSho98}, with a cost of $O(n^{1.82}\log q)$ operations in
$\FF_q$.  In the bit complexity model, Kedlaya and Umans showed how to
reduce the exponent of $n$ to $1.63$, by leveraging their quasi-linear
time algorithm for {\em modular
  composition}~\citep{KeUm11}. \cite{LenstraNormal} introduced a
deterministic algorithm to construct a normal element which uses
$n^{O(1)}$ operations in $\FF_{q^n}/\FF_q$.  To the best of our
knowledge, the algorithm of \cite{AugCam94} is the most efficient
deterministic method, with a cost of $O(n^3+n^2\log q)$ operations in
$\FF_q$.

In characteristic zero, \cite{SchSte93} gave an algorithm for finding
a normal basis of a number field over $\QQ$ with a cyclic Galois group
of cardinality $n$ which requires $n^{O(1)}$ operations in $\QQ$.
\cite{Pol94} gives an algorithm for the more general case of finding a
normal basis in an abelian extension $\K/\F$ which requires $n^{O(1)}$
operations in $\F$.  More generally in characteristic zero, for any
Galois extension $\K/\F$ of degree $n$ with Galois group given by a
collection of $n$ matrices, \cite{Girstmair} gives an algorithm which
requires $O(n^4)$ operations in $\F$ to construct a normal element in
$\K$.

In this paper we present a new randomized algorithm that decides
whether a given element in either an abelian or a metacyclic extension
is normal, with a runtime subquadratic in the degree $n$ of the
extension. The costs of all algorithms are measured by counting
\emph{arithmetic operations} in $\F$ at unit cost.  Questions related
to the bit-complexity of our algorithms are challenging, and beyond
the scope of this paper.

Our main conventions are the following.
\begin{assumption}
  \label{assum}
  Let $\K/\F$ be a finite Galois extension presented as
  $\K=\F[x]/\langle P(x)\rangle$, for an irreducible polynomial $P\in
  \F[x]$ of degree $n$, with $\F$ of characteristic zero. Then,
  \begin{itemize}
  \item elements of $\K$ are written on the power basis $1,\xbar,\dots,\xbar^{n-1}$,
    where $\xbar := x \bmod P$;
  \item elements of $G$ are represented by their action on $\xbar$.
  \end{itemize}
\end{assumption}

In particular, for $g \in G$ given by means of $\gamma:=g(\xbar) \in \K$,
and $\beta = \sum_{0\leq i<n}\beta_i\xbar^i\in\K$, the fact that $g$ is an
$\F$-automorphism implies that $g(\beta)$ is equal to $\beta(\gamma)$, the
polynomial composition of $\beta$ at $\gamma$ (reduced modulo $P$).

Our algorithms combine techniques and ideas of~\cite{GatGie90} and
\cite{KalSho98}: $\alpha \in \K$ is normal if and only if the element
$S_\alpha := \sum_{g \in G} g(\alpha)g \in \K[G]$ is invertible in the
group algebra $\K[G]$.  However, writing down $S_\alpha$ involves
$\Theta(n^2)$ elements in $\F$, which precludes a subquadratic
runtime. Instead, knowing $\alpha$, the algorithms use a randomized
reduction to a similar question in $\F[G]$, that amounts to applying a
random projection $\ell:\K\to\F$ to all entries of $S_\alpha$, giving
us an element $s_{\alpha,\ell} \in \F[G]$. For that, we adapt
algorithms from~\citep{KalSho98} that were developed for Galois groups
of finite fields.

Having $s_{\alpha,\ell}$ in hand, we need to test its
invertibility. In order to do so, we present an algorithm in the
abelian case which relies on the fact that $\F[G]$ is isomorphic to a
multivariate polynomial ring modulo an ideal $(x^{e_i}_i-1)_{1 \leq i
  \leq m}$, where $e_i$'s are positive integers. For metacyclic
groups, we exploit the block-Hankel structure of the matrix of
multiplication by $s_{\alpha,\ell}$. 

These latter questions on the cost of arithmetic operations in $\F[G]$
are closely related to that of Fourier transform over $G$, and it is
worth mentioning that there is a vast literature on fast algorithms
for Fourier transforms (over the base field $\C$). Relevant to our
current context, consider \citep{ClaMu04} and \citep{MaRockWol18} and
references therein for details. At this stage, it is not clear how we
can apply these methods in our context (where we work over an
arbitrary $\F$, not necessarily algebraically closed).

This paper is written from the point of view of obtaining improved
asymptotic complexity estimates. Since our main goal is to highlight
the exponent (in $n$) in our runtime analyses, costs are given using
the soft-O notation: $S(n)$ is in $\tilde{O}(T(n))$ if it is in
$O(T(n) \log(T(n))^c)$, for some constant $c$.

The first main result of this paper is the following theorem; we use a
constant $\omega(4/3)$ that describes the cost of certain rectangular
matrix products (see the end of this section).
\begin{theorem}\label{thm:main}
  Under Assumption \ref{assum}, if $G$ is either abelian or
  metacyclic, one can test whether $\alpha \in \K$ is normal using
  $\thecost$ operations in $\F$, with
  $(3/4)\cdot\omega(4/3)<1.99$. The algorithms are randomized.
\end{theorem}
Once $\alpha$ is known to be normal, we also discuss the cost 
of conversion between the power basis $1,\xbar,\dots,\xbar^{n-1}$
of $\K$ and its normal basis $G\cdot \alpha$. Again inspired by
previous work of~\cite{KalSho98}, we obtain the following results.
\begin{theorem}\label{thm:main2}
  Under Assumption \ref{assum}, if $G$ is either abelian or metacyclic
  and $\alpha \in \K$ is known to be normal, we can perform basis
  conversion between the power basis $1,\xbar,\dots,\xbar^{n-1}$ of
  $\K$ and its normal basis $G\cdot \alpha$ using $\thecost$
  operations in $\F$. The algorithms are randomized.
\end{theorem}
In both theorems, the runtime is barely subquadratic, and the exponent
$1.99$ is obtained through fast matrix multiplication algorithms that
are most likely impractical for reasonable $n$. However, these results
show in particular that we can perform basis conversions without
writing down the normal basis itself (which would require
$\Theta(n^2)$ elements in $\K$).

Section \ref{sec:pre} of this paper is devoted to definitions and
preliminary discussions.  In Section \ref{sec:osum}, a
subquadratic-time algorithm is presented for the randomized reduction
of our main question to invertibility testing in $\F[G]$; this
algorithm applies to any finite polycyclic group, and in particular to
abelian and metacyclic groups. In Section \ref{sec:invertibility}, we
show that the latter problem can be solved in quasi-linear time for an
abelian group; for metacyclic groups, we give a subquadratic time
algorithm based on structured linear algebra algorithms. Finally,
Section~\ref{sec:conversion} proves Theorem~\ref{thm:main2}.

Our algorithms make extensive use of known algorithms for polynomial and
matrix arithmetic; in particular, we use repeatedly the fact that
polynomials of degree $d$ in $\F[x]$, for any field $\F$ of
characteristic zero, can be multiplied in $\tilde{O}(n)$ operations in
$\F$~\citep{ScSt71}. As a result, arithmetic operations
$(+,\times,\div)$ in $\K$ can all be done using $\tilde{O}(n)$
operations in $\F$~\citep{vzGathen13}.

For matrix arithmetic, we will rely on some non-trivial results on
rectangular matrix multiplication initiated by \cite{LoRo83}. For $k \in
\mathbb{R}$, we denote by $\omega(k)$ a constant such that over any
ring, matrices of sizes $(n,n)$ by $(n,\lceil n^k \rceil)$ can be
multiplied in $O(n^{\omega(k)})$ ring operations (so $\omega(1)$ is
the usual exponent of square matrix multiplication, which we simply
write $\omega$).  The sharpest values known to date for most
rectangular formats are by~\cite{LeGall}; for $k=1$, the best known
value is $\omega \le 2.373$ by \citeN{LeGall14}. Over a field, we will
frequently use the fact that further matrix operations (determinant or
inverse) can be done in $O(n^\omega)$ base field operations.


Part of the results of this paper (Theorem~\ref{thm:main} for abelian
groups) were already published in the conference
paper~\citep{GiJaSc19}.

%%% Local Variables:
%%% mode: latex
%%% TeX-master: "NormalBasisCharZero"
%%% End:

\section{Preliminaries}
\label{sec:pre}

One of the well-known proofs of the existence of a normal element for
a finite Galois extension \cite[Theorem 6.13.1]{Lang} suggests a
randomized algorithm for finding such an element. Assume $\K/\F$ is a
finite Galois extension with Galois group $G = \lbrace g_1 , \ldots ,
g_n \rbrace$. If $x \in \K$ is a normal element, then
\begin{equation}
  \label{eq:fstrow}
  \sum_{j=1}^n 
  c_j g_j(x)=0, \,\,\, c_j \in \F 
\end{equation} 
implies $(c_1, \ldots ,c_n) = 0$. For each $i \in \lbrace 1, \ldots , n\rbrace$, applying $g_i$ to equation (\ref{eq:fstrow}) yields
\begin{equation} \label{eq:otherrow}
 \sum_{j=1}^n 
 c_j g_i g_j(x)=0.
\end{equation}
Using \eqref{eq:fstrow} and \eqref{eq:otherrow}, one can form a linear system $\mat{M}_G(\alpha)\textbf{v} = \textbf{0}$ where 
 $$ M_G(x) =
\begin{bmatrix}
g_1 g_1(x) & g_1 g_2(x) & \cdots & g_1 g_n(x) \\
g_2 g_1(x) & g_2 g_2(x) & \cdots & g_2 g_n(x) \\
\vdots		& \vdots	& \vdots & \vdots \\
g_n g_1(x) & g_n g_2(x) & \cdots & g_n g_n(x) \\
\end{bmatrix}. 
 $$
 Equation \ref{eq:otherrow} shows that $M_G(x)$ is non-singular. The rest of the proof shows that there exists $\alpha \in \K$ with $\det(M_G(\alpha))\neq 0$.
 
 
 Above discussion can be used as the basic idea of a randomized algorithm for finding a
 normal element.\\
 \\
 \textbf{Algorithm 1.} \label{alg:naive}
 The algorithm takes a finite Galois extension $\K/\F$ with 
 $G =  \mathrm{Gal}(\K/\F) = \lbrace g_1, g_2, \ldots , g_n \rbrace$ and returns a normal
 element $\alpha \in \K$.
 \begin{description}
 \item \textbf{step 1.} choose a random element $\alpha$ in $\K$.
 \item \textbf{step 2.} write the matrix $M_G(\alpha)$.
 \item \textbf{step 3.} if $ M_G(\alpha)$ is invertible\\
% \hspace{2cm} if $r = n$ \\
 \hspace{10cm} return $\alpha$\\
 \hspace{2cm} else \\
 \hspace{5cm} go to step 1.\\  
 \end{description}
 
 Clearly the main concerns are step 2 and step 3. As a naive way, one can compute all the entries of the matrix and then 
 use linear algebra to compute the determinant of $M_G(x)$ which uses $O(n^3)$ operations. However, since this is not efficient
 enough, we want to avoid writing the matrix and computing determinants. Before explaining how we can check the invertiblity 
 in an efficient way,it is worth talking about the probability of success for a random choice of $\alpha$.
 
 Since we are working over characteristic zero, we are able to consider a finite subset
 of $\K \cong \F^n$ such that the probability of failure is $\dfrac{1}{n}$, using 
 Schwartz-Zippel lemma (Proposition \ref{thm:zippel}).
 
 If $\lbrace a_1, \ldots , a_n \rbrace$ is an $\F$-basis for $\K$, then in Equations
   \ref{eq:fstrow} and \ref{eq:otherrow}, $x$ can be written as $\sum_{i = 1}^nx_i
    a_i$. Now we can rewrite 
    $$
M_G(x) = M_G(x_1,\ldots,x_n) =  $$
$$
\begin{bmatrix}
\sum_{i = 1}^n g_1 g_1(a_i)x_i & \sum_{i = 1}^n g_1 g_2(a_i)x_i & \cdots & 
\sum_{i = 1}^n g_1 g_n(a_i)x_i \\
\sum_{i = 1}^n g_2 g_1(a_i)x_i & \sum_{i = 1}^n g_2 g_2(a_i)x_i & \cdots & 
\sum_{i = 1}^n g_2 g_n(a_i)x_i \\
\vdots		& \vdots	& \vdots & \vdots \\
\sum_{i = 1}^n g_n g_1(a_i)x_i & \sum_{i = 1}^n g_n g_2(a_i)x_i & \cdots & 
\sum_{i = 1}^n g_n g_n(a_i)x_i \\
\end{bmatrix}    
    $$ 
 The following theorem enables us to talk about probability of success (or failure).
 \begin{proposition}\cite[Proposition 98]{Zippel} \label{thm:zippel}
Let $P \in A[X_1, \ldots, X_n]$ be a polynomial with total degree $D$ over an integral domain $A$. Let $S$ be a subset of $A$ of cardinality $B$. Then $$Pr(P(x_1, \ldots , x_n)=0:x_i \in S) \leq \dfrac{D}{B}.$$
\end{proposition}

Now $\det(M_G(x)) \in \K[x_1, \ldots , x_n]$ is a polynomial of total degree $n$ and $\K$ is a field. If $char(\K) =0$, then by considering $S \subset \F \subset \K$ where $|S| = n^2$ and applying the above proposition we get $$Pr(\det(M_G(r_1,\ldots , r_n)) = 0 : r_i \in S)\leq \dfrac{n}{n^2}= \dfrac{1}{n}.$$
 
Now let us take a look at steps 2 and 3 in Algorithm \ref{alg:naive}. Since in the cyclic case $M_G(\alpha)$ is circulant,
 in \cite{Giesbrecht} instead of writing the matrix, the invertibility test is done by means of finding the gcd of an 
 associated polynomial 
$$P(x) = \sum_{i = 1}^n g_ig_1(\alpha)x^{i-1}$$ 
 to $M_G(\alpha)$ and $x^n-1$. Note that in this case the group ring of $G$ over $\K$, $\K[G]$ is isomorphic to $\K[x]/(x^n-1)$,
 $P(x)$ is the isomorphic copy of the orbit sum of $\alpha$ over $\K[G]$, namely
 $$\alpha_{G,\K} = \sum_{g \in G} g(\alpha)g \in \K[G].$$
 Moreover, invertibility of $\osum{G}{\K}\in \K[G]$ is equivalent of having $\gcd (P(x),x^n-1) =1.$
 
 A close look at $M_G(x)$ tells us it is exactly the matrix of (left) multiplication by $$\osum{G}{\K} \in \K[G].$$  This gives an idea to modify Algorithm \ref{Alg:Naive}. Instead of writing $M_G(x)$ and test its  invertiblity, 
 we can write $\osum{G}{\K}$ and test if it is invertible in $\K[G]$. Although testing the invertibility of $\osum{G}{\K}$ might be efficient in
 comparison to computing the determinant of a matrix in $\K$, we prefer to do the computations over $\F$ rather than $\K$. The  following lemma
 comes handy to pass the computations from $\K$ to $\F$.


\begin{lemma}\label{Lem:Proj}
Assume $\alpha \in \K$. $M_G(\alpha) \in M_{n \times n}(\K)$ 
is invertible if and only if $$l(M_G(\alpha)) =  [l(g_ig_j(\alpha))]_{ij}  \in M_{n \times n}(\F)$$ is invertible,
 where $l$ is a generic projection of $\K$ to $\F$.
\end{lemma}

\begin{proof}
$(\Rightarrow)$ Since $\K = \F(\theta)$, for a fixed $\alpha$, any entry of $M_G(\alpha)$ can be written as 
\begin{equation}\label{Eq:PrimElm}
\sum_{k= 0}^{n-1} a_{ijk}\theta^k
\end{equation}
 and the corresponding entry in $l(M_G(\alpha))$ (for a random projection $l$)
 can be written $\sum_{k= 0}^{n-1} a_{ijk}l_k$ with $l_k\in \F$. If we replace this specific choice of $l_k$'s by 
 indeterminates $x_k$'s, we can see $\det(x(M_G(\alpha))$ is a polynomial in $\F[x_1, \ldots, x_n].$ Let 
 $$P(x_1, \ldots, x_n) = \det(x(M_G(\alpha)).$$ 
 Considering $P \in \K[x_1, \ldots , x_n]$, one can verify that 
 \begin{equation}\label{Eq:Det}
 det(M_G(\alpha))= P(1, \theta, \ldots, \theta^{n-1}) \neq 0
 \end{equation}
 since $M_G(\alpha)$ is invertible. Equation \ref{Eq:Det} implies that $P(x_1, \ldots, x_n)$ is not identically zero over $\F$. Hence applying Theorem \ref{Thm:Zippel} with appropriate choice of parameters, we can see 
 the projection of $M_G(\alpha)$ is invertible for a generic choice of projection. 
 
 $(\Leftarrow)$  Note that elements of $G$ can act on 
 rows of $M_G(\alpha)$ entrywise and the action permutes the rows of $M_G(\alpha)$. Assume $\varphi : G \longrightarrow \mathfrak{S}_n$ is the group homomorphism 
 such that $g(M_i) = M_{\varphi(g)(i)}$ where $M_i$ is the $i$-th row of $M_G(\alpha)$.
 
 Assume $M_G(\alpha)$ is not invertible. Following the proof of \cite[Lemma 4]{Armin}, we show that there exists a non-zero $\textbf{u} \in \F^n$ in the kernel of $M_G(\alpha)$. 
 
 Since $M_G(\alpha)$ is singular, there exists a non-zero $\textbf{v} \in \K^n$  such that $M_G(\alpha)\textbf{v} = 0$ and $\textbf{v}$ has the minimum number of non-zero entries. Let $i \in  \lbrace 1, \ldots , n \rbrace$ such that $v_i \neq 0$. Define $\textbf{u} = \dfrac{1}{v_i}\textbf{v}$. It is clear  that $M_G(\alpha)\textbf{u} = 0$ which means $M_j \textbf{u} = 0 $ for $j \in \lbrace 1, \ldots, n \rbrace$. For $g \in G$
 \begin{equation}
  g(M_j \textbf{u}) = M_{\varphi(g)(j)} \textbf{u}= 0
 \end{equation}
 Since the above equation holds for any $j$ we conclude that $$M_G(\alpha)g(\textbf{u})= 0$$ hence
 $g(\textbf{u})-\textbf{u}$ is in the kernel of $M_G(\alpha)$. On the other hand since the $i$-th entry 
 of $\textbf{u}$ is one, the $i$-th entry of $g(\textbf{u}) -\textbf{u}$ is zero. Thus the minimality assumption
 on $\textbf{v}$ shows that $g(\textbf{u}) -\textbf{u} = 0$ and equivalently $g(\textbf{u})=\textbf{u}$. This 
 means $\textbf{u} \in \F^n$.
 
 
 Now we show that $l(M_G(\alpha))$ is not invertible for all
 choices of $l$. By Equation \ref{Eq:PrimElm} we can write 
 $$M_G(\alpha) = \sum_{j = 1}^n M^{(j)} \theta^j$$ 
 where $M^{j} \in M_{n \times n}(\F)$. 
 
 Now $M_G(\alpha) \textbf{u} =0$ yields $M^{(j)}\textbf{u} = 0$ for $j \in \lbrace 1, \ldots , n \rbrace$. Hence
 $$\sum_{j = 1}^n M^{(j)} l_j \textbf{u} = 0$$ for any choice of $l_j$'s in $\F$. So $l(M_G(\alpha))$ is not invertible for any choice of $l$.
\end{proof} 
Lemma \ref{Lem:Proj} enables us to test invertibility of a random projection of $l(\alpha_{G,\K}) $ i.e. $\sum_{g \in G}
 l(g(\alpha))g \in \F[G]$. Although we can avoid writing $M_G(\alpha)$, we still need to compute $l(\alpha_{G,\K})$ which
 we call it the orbit sum projection problem.

%%% Local Variables:
%%% mode: latex
%%% TeX-master: "NormalBasisCharZero"
%%% End:

\section{Computing projections of the orbit sum}
\label{sec:osum}

In this section we present algorithms to compute $s_{\alpha,\ell}$,
when $G$ is either abelian or metacyclic. We start by sketching our
ideas in simplest case, cyclic groups.  We will see that they follow
closely ideas used in \cite{KalSho98} over finite fields.

Suppose $G = \langle g \rangle$, so that given $\alpha$ in $\K$ and
$\ell: \K \to \F$, our goal is to compute
\begin{equation}
  \label{eq:cycproj}
  \ell(g^i(\alpha)), ~~\mbox{for}~ 0\leq i\leq n-1.
\end{equation}
\citeN{KalSho98} call this the \emph{automorphism projection problem} and
gave an algorithm to solve it in subquadratic time, when $g$ is the
$q$-power Frobenius $\mathbb{F}_{q^n} \to \mathbb{F}_{q^n}$.  The key idea in their
algorithm is to use the baby-steps/giant-steps technique: for a suitable
parameter $t$, the values in \eqref{eq:cycproj} can be rewritten as
\[
  (\ell \circ g^{tj})(g^i(\alpha)), ~~\mbox{for}~ 0 \leq j < m:=\lceil n/t
  \rceil ~\mbox{and}~ 0 \leq i <t.
\]
First, we compute all $G_i:=g^i(\alpha)$ for $0 \leq i <t$.  Then we compute
all $L_j:=\ell \circ g^{tj}$ for $0 \leq j <m$, where the $L_j$'s are
themselves linear mappings $\K \to \F$.  Finally, a matrix product yields
all values $L_j(G_i)$.

 
The original algorithm of \citeN{KalSho98} relies on the properties of the
Frobenius mapping to achieve subquadratic runtime. In our case, we cannot
apply these results directly; instead, we have to revisit the proofs
of~\citeN{KalSho98}, Lemmata 3, 4 and~8, now considering rectangular matrix
multiplication.  Our exponents involve the constant $\omega(4/3)$, for
which we have the upper bound $\omega(4/3) < 2.654$: this follows from the
upper bounds on $\omega(1.3)$ and $\omega(1.4)$ given by~\citeN{LeGall}, and
the fact that $k \mapsto \omega(k)$ is convex~\cite{LoRo83}. In particular,
$3/4 \cdot \omega(4/3) < 1.99$. Note also the inequality
$\omega(k) \ge 1+k$ for $k\ge 1$, since $\omega(k)$ describes products with
input and output size $O(n^{1+k})$.

%%%%%%%%%%%%%%%%%%%%%%%%%%%%%%%%%%%%%%%%%%%%%%%%%%%%%%%%%%%%

\subsection{Multiple automorphim evaluation and applications}

The key to the algorithms below is the remark following
Assumption~\ref{assum}, which reduces automorphism evaluation to
composition of polynomials.  Over finite fields, this idea goes back
to~\citeN{GaSh92}, where it was credited to Kaltofen.

For instance, given $g \in G$ (by means of $\gamma:=g(\bar x)$), we can
deduce $g^2 \in G$ (again, by means of its image at $\bar x$) as
$\gamma(\gamma)$; this can be done in $\tilde{O}(n^{(\omega+1)/2})$
operations in $\F$ using Brent and Kung's modular composition
algorithm~\cite{BrKu78}. The algorithms below describe similar operations 
along these lines, involving several simultaneous evaluations.

\begin{lemma}\label{lem:modcom}
  Given $\alpha_1,\dots,\alpha_s$ in $\K$ and $g$ in $G =
  \mathrm{Gal}(\K/\F)$, with $s = O(\sqrt{n})$, we can compute
  $g(\alpha_1),\dots,g(\alpha_s)$ in time $\tilde
  O(n^{{3}/{4}\omega({4}/{3})})$.
\end{lemma}
\begin{proof}
(Compare \cite[Lemma~3]{KalSho98}.) As noted above, for $i\le s$,
  $g(\alpha_i) = \alpha_i(\gamma)$, with $\gamma := g(\bar x) \in \K$.
  Let $t := \lceil n^{3/4} \rceil$, $m:=\lceil n/t\rceil$, and rewrite $\alpha_1 , \ldots , \alpha_s$ as 
$$\alpha_i = \sum_{0 \leq j < m} a_{i,j}\bar x^{tj},$$ where the
  $a_{i,j}$'s are polynomials of degree less than $t$. The next step
  is to compute $\gamma_i := \gamma^i$, for $i = 0 , \ldots , t$;
  there are $t$ products in $\K$ to do, so this amounts to
  $\tilde{O}(n^{7/4})$ operations in $\F$.

  Having $\gamma_i$'s in hand, one can form the matrix
  $\boldsymbol{\Gamma} := \left[ \Gamma_0 ~ \cdots ~ \Gamma_{t-1}
    \right]^T$, where each column $\Gamma_i$ is the coefficient vector
  of $\gamma_i$ (with entries in $\F$); this matrix has $t \in
  O(n^{3/4})$ rows and $n$ columns. We also form
  $$\mat A := \left[{A}_{1,0} \cdots {A}_{1,m-1} \cdots
    {A}_{s,0} \cdots {A}_{s,m-1}\right]^T,$$ where
  ${A}_{i,j}$ is the coefficient vector of $a_{i,j}$. This matrix 
  has $s m \in O(n^{3/4})$ rows and $t \in O(n^{3/4})$ columns.

  Compute $\mat B:=\mat A\, \boldsymbol{\Gamma}$; as per our
  definition of exponents $\omega(\ .\ )$, this can be done in
  $O(n^{3/4 \omega(4/3)})$ operations in $\F$, and the rows of this matrix
  give all $a_{i,j}(\gamma)$.  The last step to get all
  $\alpha_i(\gamma)$ is to write them as $\alpha_i(\gamma) = \sum_{0
    \leq j < m} a_{i,j}(\gamma) \Gamma_t^{j}.$ Using Horner's scheme,
  this takes $O(sm)$ operations in $\K$, which is $\tilde{O}(n^{7/4})$
  operations in $\F$. Since we pointed out that $\omega(3/4) \ge 7/4$,
  the leading exponent in all costs seen so far is
  ${3}/{4}\omega({4}/{3})$.
\end{proof}

\begin{lemma}\label{lem:selfcomp}
Given $\alpha$ in $\K$, $g_1, \ldots , g_{r}$ in $G =
\mathrm{Gal}(\K/\F)$ and positive integers $(s_1, \ldots s_r)$ such
that $\prod_{i = 1}^r s_i = O(\sqrt{n})$ and $r \in O(\log(n))$, all
  $$g_1^{i_1}\cdots g_r^{i_r}(\alpha) ,\quad \text{~for~} 0 \leq i_j
\leq s_j,\ 1 \leq j \leq r$$ can be computed in $\osumcosttilde$
operations in $\F$.
\end{lemma}
\begin{proof}
(Compare \cite[Lemma~4]{KalSho98}.) For $m=1,\dots,r$, suppose we have computed 
  $$G_{i_1,\dots,i_m}:=g_m^{i_m}\cdots g_1^{i_1}(\alpha)$$ for $0 \leq
  i_j \leq s_j$ if $1 \leq j < m$, and $0 \leq i_m < k_m,$ as well as
  the automorphism $\eta:={g_m}^{k_m}$ (by means of its value at $\bar
  x$, as per our convention).
  
 Then, we can get $G_{i_1,\dots,i_m}$ for $0 \leq i_j \leq s_j$ if $1
 \leq j < m$, and $0 \leq i_m < 2k_m$, by computing
 $\eta(G_{i_1,\dots,i_m})$, for all indices $i_1,\dots,i_m$ available
 to us, that is, $0 \leq i_j \leq s_j$ if $1 \leq j < m$, and $0 \leq
 i_m < k_m$. This can be carried out using $\osumcosttilde$ operations
 in $\F$ by applying Lemma \ref{lem:modcom}. Prior to entering the
 next iteration, we also compute $\eta^2$ by means of one modular
 composition, whose cost is negligible. 

 Using the above doubling method for $g_m$, we have to do $O(\log
 s_m)$ steps, for a total cost of $\osumcosttilde$ operations in $\F$.  We
 repeat this procedure for $m=1,\dots,r$; since $r$ is in $O(\log(n))$,
 the cost remains $\osumcosttilde$.
\end{proof}

We now present dual versions of the previous two lemmas (our
reference~\cite{Kaltofen} also had such a discussion). Seen as an
$\F$-linear map, the operator $g:\alpha \mapsto g(\alpha)$ admits a
transpose, which maps an $\F$-linear form $\ell:\K\to\F$ to the
$\F$-linear form $\ell \circ g: \alpha \mapsto \ell(g(\alpha))$.  The
{\em transposition principle}~\cite{KaKiBs88,CaKaYa89} implies that if
a linear map $\F^N \to \F^M$ can be computed in time $T$, its
transpose can be computed in time $T+O(N+M)$. In particular, given $s$
linear forms $\ell_1,\dots,\ell_s$ and $g$ in $G$, transposing
Lemma~\ref{lem:modcom} shows that we can compute $\ell_1 \circ
g,\dots,\ell_s \circ g$ in time $\osumcosttilde$. The following lemma
sketches the construction.

\begin{lemma}\label{lem:modcomT}
  Given $\F$-linear forms $\ell_1,\dots,\ell_s:\K\to \F$ and $g$ in $G =
  \mathrm{Gal}(\K/\F)$, with $s = O(\sqrt{n})$, we can compute
  $\ell_1\circ g,\dots,\ell_s \circ g$ in time $\tilde
  O(n^{{3}/{4}\omega({4}/{3})})$.
\end{lemma}
\begin{proof}
  Given $\ell_i$ by its values on the power basis $1,\bar x,\dots,\bar
  x^{n-1}$, $\ell_i \circ g$ is represented by its values at
  $1,\gamma,\dots,\gamma^{n-1}$, with $\gamma := g(\bar x)$. 

  Let then $t,m$ and $\gamma_0,\dots,\gamma_t$ be as in the proof of
  Lemma~\ref{lem:modcom}. Next, compute the ``giant steps''
  $\gamma_t^j = \gamma^{tj}$, $j=0,\dots,m-1$ and for $i=1,\dots,s$
  and $j=0,\dots,m-1$, deduce the linear forms $L_{i,j}$ defined by
  $L_{i,j}(\alpha) := \ell_i(\gamma^{tj}\alpha)$ for all $\alpha$ in
  $\K$. Each of them can be obtained by a {\em transposed
    multiplication} in time $\tilde{O}(n)$~\cite[Section~4.1]{Shoup},
  so that the total cost thus far is $\tilde{O}(n^{7/4})$.

  Finally, multiply the $(sm \times
  n)$ matrix with entries the coefficients of all $L_{i,j}$ (as rows)
  by the $(n \times t)$ matrix with entries the coefficients of
  $\gamma_0,\dots,\gamma_{t-1}$ (as columns) to obtain all values
  $\ell_i(\gamma^j)$, for $i=1,\dots,s$ an $j=0,\dots,n-1$.
  This takes $O(n^{{3}/{4}\omega({4}/{3})})$ operations in~$\F$.
\end{proof}

From this, we deduce the transposed version of Lemma~\ref{lem:selfcomp},
whose proof follows the same pattern.
\begin{lemma}\label{lem:transmodcomp}
Given $\ell:\K\to F$, $g_1, \ldots , g_{r}$ in $G =
\mathrm{Gal}(\K/\F)$ and positive integers $(s_1, \ldots s_r)$ such
that $\prod_{i = 1}^r s_i = O(\sqrt{n})$ and $r \in O(\log(n))$, all
linear maps
  $$\ell \circ g_1^{i_1}\cdots g_r^{i_r} ,\quad \text{~for~} 0 \leq i_j
\leq s_j,\ 1 \leq j \leq r$$ can be computed in $\osumcosttilde$
operations in $\F$.
\end{lemma} 
\begin{proof}
  We proceed as in Lemma~\ref{lem:selfcomp}. For $m=1,\dots,r$,
  assumie we know
  $L_{i_1,\dots,i_m}:=\ell \circ (g_1^{i_1}\cdots g_m^{i_m}),$ for
  $0 \leq i_j \leq s_j$ if $1 \leq j < m$, and $0 \leq i_m < k_m.$
  Using the previous lemma, we compute all $L_{i_1,\dots,i_m} \circ
  {g_m}^{k_m},$ which gives us $L_{i_1,\dots,i_m}$ for indices $0 \le
  i_m < 2k_m$. The cost analysis is as in Lemma~\ref{lem:selfcomp}.
\end{proof}

%% At this point we have enough tools to see how the computation is
%% done in the cyclic case. Moreover, we can use the above lemmas to
%% give an algorithm for a more general case, namely the abelian
%% case. With a little bit more work we can state an algorithm which
%% solves the automorphism projection problem when $G = \lbrace a^ib^j
%% \rbrace$ where $m \leq n$. As an specific case this solves the
%% problem for metacyclic groups.

%%%%%%%%%%%%%%%%%%%%%%%%%%%%%%%%%%%%%%%%%%%%%%%%%%%%%%%%%%%%

\subsection{Abelian Groups}\label{ssec:proj_abelian}

Assume $G$ is an abelian group presented as 
$$ \langle g_1, \ldots , g_r: g_{1}^{e_1} = \cdots = g_{r}^{e_r} = 1 \rangle$$
 where $ e_i \in \mathbb{N}$
is the order of $g_i$ and $n = e_1 \cdots e_r$. Moreover, let $s_i = \lceil	\sqrt{e_i \rceil}$ for $ 1\leq i \leq r$.
Our goal is to compute 

\begin{equation}\label{eq:abelian}
L (g_1^{i_1},  \ldots, g_r^{i_r}(\alpha)), \, 1 \leq j \leq r, 0 \leq i_j \leq e_j
\end{equation}
 where $L = \begin{bmatrix} l_1 & \cdots l_n \end{bmatrix}$ is a given projection from $K$ to $F$. 
similar to the cyclic case, the elements in \eqref{eq:abelian} can be presented as 
\begin{equation}
\begin{split}
(L \circ g_1^{s_1j_1} \cdots g_r^{s_rj_s})\cdot (g_1^{i_1} \cdots g_r^{i_r}(\alpha))\\ 1\leq m \leq r, 0\leq i_m < s_m, 0 \leq j_m < s_m
\end{split}
\end{equation}
note that $T_{j_1\cdots j_r} = L \circ g_1^{s_1j_1} \cdots g_r^{s_rj_s}$ are linear projections 
presented as row vectors and $g_1^{i_1} \cdots g_r^{i_r}(\alpha)$ are field elements presented as column vectors. All elements in \eqref{eq:abelian} can be computed in 3 steps.

\textbf{Step 1.} Apply Lemma \ref{lem:selfcomp} to get 
$$g_1^{i_1} \cdots g_r^{i_r}(\alpha), \,\, 1\leq m \leq r, 0\leq i_m < s_m,$$
with cost $\osumcosttilde$

\textbf{Step 2.} Use Lemma \ref{lem:transmodcomp} to compute 
$$T_{j_1\cdots j_r}, \,\, 1\leq m \leq r, 0 \leq j_m < s_m,$$
with cost $\osumcosttilde$

\textbf{Step 3.} The last step is to the following matrix multiplication

$$
\left[ \begin{array}{c}
T_{00\cdots 0}\\
\hline
T_{10\cdots 0}\\
\hline
\vdots\\
\hline
T_{s_1 \cdots s_r}
\end{array} \right]
\cdot
\left[\begin{array}{c}
g_1^{0}\cdots g_r^{0}(\alpha) \\
\hline
g_1^{1}\cdots g_r^{0}(\alpha) \\
\hline
\cdots \\
\hline
g_1^{s_1}\cdots g_r^{s_r}(\alpha)  
\end{array}\right]^t.
$$
Using results for rectangular matrix multiplication we can compute the above product in $O(n^{1/2\omega(2)})$ operations in $F$.
The above computation can be done in $\osumcosttilde$ operations in $F$ which produces a $\lceil \sqrt{n} \rceil \times \lceil
 \sqrt{n} \rceil$ matrix which contains all the projections. Concatenating the rows of the mentioned matrix forms a vector in
 $F[G]$ which is the corresponding vector to $\osum{G}{F}$ with respect to the basis $G$. This means we have proved the following
 proposition.

\begin{proposition}
Suppose Assumption \ref{assum} holds and $G$ is an abelian group. $l(\osum{K}{G}) \in F[G]$ is computable using $\thecost$ 
operations in $F$.
\end{proposition}

\subsection{Metacyclic Groups}

A Group $G$ is called metacyclic if it has a normal cyclic subgroup, $H$, such that $G/H$ is cyclic. It is known that any group
with a square free order, is metacyclic and elements of a metacyclic group can be presented as 
\begin{equation}\label{eq:metacyclic}
\langle \sigma,\tau: \sigma^n = 1, \tau^{-1}\sigma \tau = \sigma^r, \tau ^m = \sigma^s \rangle
\end{equation}
where $m,n,r,s \in \mathbb{N}, r,s \leq n,$ and $r^m = 1 \mod n , rs = s \mod n$. Moreover we know that all element of a metacyclic
 group can be presented by $$\sigma^i \tau^j, \,\,\, 0\leq i \leq m-1, 0\leq j \leq n-1.$$ 
for more details on metacyclic groups see \cite[P.88, Proposition 1]{Johnson}, \cite[P.334]{Curtis}. for constructing some 
Metacyclic extensions see \cite{Kida}.

. Dihedral group 
$$D_{2n} = \langle \sigma,\tau: \sigma^n =\tau^2 = 1, \sigma \tau = \tau \sigma^{-1} \rangle, $$
is an example of metacyclic groups. another well-known metacyclic group is generalized quaternion
 group which can be presented as
 $$Q_n = \langle \sigma,\tau: \sigma^n =\tau^2, \tau \sigma \tau^{-1} = \sigma^{-1} \rangle.$$
 We know that elements of $Q_n$ are of the form 
 $$\sigma^i\tau^j, 0 \leq i \leq 2n-1 , 0\leq j \leq 1.$$
 
Assume $G = \mathrm{Gal(K/F})$ is a group presentable as
$$G = \lbrace \sigma^i \tau^j: 0\leq i < n, 0 \leq j < m, m\leq n \rbrace,$$
and $\alpha\in K$. The goal is to compute 
$L(\sigma^i\tau^j (\alpha), \,\, 0\leq i < n, 0 \leq j <m.$
This can be done in three steps.

\textbf{step 1.} apply Lemma \ref{lem:selfcomp} to compute 
$$s_{ij} = \sigma^i\tau^j(\alpha), 0 \leq j < m, 0\leq i < \lceil \sqrt{n}/\sqrt{m} \rceil$$
note that $\lceil \sqrt{n}/\sqrt{m} \rceil m \leq \lceil \sqrt{mn} \rceil$.

\textbf{step 2.} compute $$T_j = L \circ \sigma^{j\sqrt{mn}}, \,\, 0\leq j < \lceil \sqrt{mn}\rceil$$
using Lemma	\ref{lem:transmodcomp}.

\textbf{step 3.} at this point we want to compute 
$$L(\sigma^i\tau^j(\alpha)) = T_k\cdot(\sigma^i(\alpha_j)).$$
This can be carried out with a rectangular matrix multiplication
$$
\left[ \begin{array}{c}
T_0\\
\hline
T_1\\
\hline
\vdots\\
\hline
T_{\lceil \sqrt{mn} \rceil-1}
\end{array} 
\right]
\cdot
\left[\begin{array}{l}
s_{00} \\
\hline
 \vdots \\
 \hline
s_{0m} \\
 \hline
\vdots \\
\hline
s_{\lceil \sqrt{n}/\sqrt{m} \rceil 0} \\
\hline
\vdots \\
\hline
s_{\lceil \sqrt{n}/\sqrt{m} \rceil m}
\end{array}
\right]^t
$$
which is a $\langle \sqrt{mn},n,\sqrt{mn}\rangle$ multiplication. 

We note the above algorithm works for a class of groups which includes metacyclic case. Since if $G$ is metacyclic, 
$H = \langle \sigma \rangle \unlhd G$ and $G/H = \langle \tau H \rangle$, then both $\tau \sigma \tau^{-1}$ and 
$\tau^{-1} \sigma \tau$ belong to $H$. This implies elements of $G$ can be presented either as $\sigma^i\tau^j$
or $\tau^j\sigma^i$. Thus without loss of generality we can assume the $\textrm{Ord}(\tau) \leq \textrm{Ord}(\sigma)$.

Similar to the abelian case the final output of the above algorithm is a $\lceil \sqrt{n} \rceil \times \lceil \sqrt{n} \rceil $
matrix and $l(\osum{G}{K})$ can be calculated in the same way. Hence we have proved the following proposition.
 
\begin{proposition}
Suppose Assumption \ref{assum} holds and $G$ is a metacyclic group. $l(\osum{K}{G}) \in F[G]$ is computable using $\thecost$ 
operations in $F$.
\end{proposition}

%%% Local Variables:
%%% mode: latex
%%% TeX-master: "NormalBasisCharZero"
%%% End:

\section{Testing Invertibility}
%in the Group Algebra}
\label{sec:invertibility}

In this section we consider the problem of invertibility testing in
$\F[G]$, specifically for abelian and metacyclic groups $G$: given an
element $\beta$ in $\F[G]$, for a field $\F$ and a group $G$, determine
whether $\beta$ is a unit in $\F[G]$.  
Since we are in characteristic zero, Wedderburn's theorem implies the
existence of an $\F$-algebra isomorphism (which we will refer to as a
Fourier Transform)
\[
  \F[G] \to M_{d_1}(D_1) \times \dots \times M_{d_r}(D_r),
\]
where all $D_i$'s are division algebras over $\F$. If we were working over
$\F=\C$, all $D_i$'s would simply be $\C$ itself.  A natural solution
to test the invertibility of $\beta \in \F[G]$ would then be to compute its
Fourier transform and test whether all its components
$\beta_1 \in M_{d_1}(\C),\dots,\beta_r \in M_{d_r}(\C)$ are
invertible. This boils down to linear algebra over $\C$, and takes
$O(d_1^\omega + \cdots + d_r^\omega)$ operations.  Since
$d_1^2 + \cdots + d_r^2 = |G|$, this is $O(|G|^{\omega/2})$ operations in
$\C$.

However, we do not wish to make such a strong assumption as $\F=\C$. Since
we measure the cost of our algorithms in $\F$-operations, the direct
approach that embeds $\F[G]$ into $\C[G]$ does not make it possible to
obtain a subquadratic cost in general. If, for instance, $\F=\Q$ and $G$ is
cyclic of order $n=2^k$, computing the Fourier Transform of $\beta$
requires we work in a degree $n/2$ extension of $\Q$, implying a quadratic
runtime.

We give algorithms for the problem of invertibility testing for the
families of group seen so far, abelian and metacyclic. For the former,
we prove a stronger result: starting from a suitable presentation of
$G$, we give a softly linear-time algorithm to find an isomorphic
image of $\beta \in \F[G]$ in a product of $\F$-algebras of the form
$\F[z]/\langle P_i(z)\rangle$, for certain polynomials $P_i \in \F[z]$
(recovering $\beta$ from its image is softly-linear time as well). Not
only does this allow us to test whether $\beta$ is invertible, this
would also make it possible to find its inverse in $\F[G]$ in
softly-linear time.  For metacyclic groups, we describe an injective
$\F$-algebra homomorphism from $\F[G]$ to a matrix algebras over a
cyclotomic ring. The codomain is in general of dimension higher than
$|G|$, so the algorithm we deduce from this is not linear-time.

%%%%%%%%%%%%%%%%%%%%%%%%%%%%%%%%%%%%%%%%%%%%%%%%%%%%%%%%%%%%

\subsection{Abelian groups}

Because an abelian group is a product of cyclic groups, it group
algebra $\F[G]$ is the tensor product of cyclic algebras, so it admits
a description of the form $\F[x_1,\dots,x_t]/\langle
x_1^{n_1}-1,\dots,x_t^{n_t}-1\rangle$, for some integers
$n_1,\dots,n_t$. The complexity of arithmetic operations in an
$\F$-algebra such as $\A:=\F[x_1,\dots,x_t]/\langle
P_1(x_1),\dots,P_t(x_t)\rangle$ is difficult to pin down
precisely. For general $P_i$'s, the cost of multiplication in $\A$ is
known to be $O(\dim(\A)^{1+\varepsilon})$, for any $\varepsilon >
0$~\cite[Theorem~2]{LiMoSc09}. From this it may be possible to deduce
similar upper bounds on the complexity of invertibility tests,
following~\cite{DaMMMScXi06}, but this seems non-trivial.

Instead, we give an algorithm with softly linear runtime, that uses
the factorization properties of cyclotomic polynomials and Chinese
remaindering techniques to transform our problem into that of
invertibility testing in algebras of the form $\F[z]/\langle P_i(z)
\rangle$, for various polynomials $P_i$. The reference~\cite{Pol94}
also discusses the factors of algebras such as
$\F[x_1,\dots,x_t]/\langle x_1^{n_1}-1,\dots,x_t^{n_t}-1\rangle$, but
the resulting algorithms are different (and the cost of the
\citeauthor{Pol94}'s \citeyear{Pol94} algorithm is only known to be
polynomial in $|G|$).

\smallskip

\noindent{\bf Tensor product of two cyclotomic rings: coprime orders.}
The following proposition will be the key to foregoing multivariate
polynomials, and replacing them by univariate ones.  Let $m,m'$ be two
coprime integers and define
$$\mathbbm{h}:=\F[x,x']/\langle \Phi_{m}(x), \Phi_{m'}(x')\rangle,$$
where for $i \ge 0$, $\Phi_i$ is the cyclotomic polynomial of order
$i$. In what follows, $\varphi$ is Euler's totient function, so that
$\varphi(i) = \deg(\Phi_i)$ for all~$i$.
\begin{lemma}
  There exists an $\F$-algebra isomorphism $\gamma: \mathbbm{h} \to
  \F[z]/\langle\Phi_{mm'}(z)\rangle$ given by $xx' \mapsto z$.  Given
  $\Phi_m$ and $\Phi_{m'}$, $\Phi_{mm'}$ can be computed in time
  $\tilde{O}(\varphi(mm'))$; given these polynomials, one can
  apply $\gamma$ and its inverse to any input using
  $\tilde{O}(\varphi(mm'))$ operations in~$\F$.
\end{lemma}
\noindent {\sc Proof.}
  Without loss of generality, we prove the first claim over $\Q$; the
  result over $\F$ follows by scalar extension. In the field \sloppy
  $\Q[x,x']/\langle \Phi_{m}(x), \Phi_{m'}(x')\rangle$, $xx'$ is
  cancelled by $\Phi_{mm'}$. Since this polynomial is irreducible, it
  is the minimal polynomial of $xx'$, which is thus a primitive
  element for $\Q[x,x']/\langle \Phi_{m}(x),
  \Phi_{m'}(x')\rangle$. This proves the first claim.

  For the second claim, we first determine the images of $x$ and $x'$
  by $\gamma$. Start from a B\'ezout relation $am+ a'm'=1$, for some
  $a,a'$ in $\Z$.  Since $x^m = {x'}^{m'}=1$ in $\mathbbm{h}$, we
  deduce that $\gamma(x)=z^{u}$ and $\gamma(x') = z^{v}$, with $u:=am
  \bmod mm'$ and $v:=a'm' \bmod mm'$. To compute $\gamma(P)$, for some
  $P$ in $\mathbbm{h}$, we first compute $P(z^u, z^v)$, keeping all
  exponents reduced modulo $mm'$. This requires no arithmetic
  operations and results in a polynomial $\bar P$ of degree less than
  $mm'$, which we eventually reduce modulo $\Phi_{mm'}$ (the latter is
  obtained by the composed product algorithm of~\cite{BoFlSaSc06} in
  quasi-linear time).  By~\cite[Theorem~8.8.7]{BacSha96}, we have the
  bound $s \in O(\varphi(s) \log(\log(s)))$, so that $s$ is in
  $\tilde{O}(\varphi(s))$. Thus, we can reduce $\bar P$ modulo
  $\Phi_{mm'}$ in $\tilde{O}(\varphi(mm'))$ operations, establishing
  the cost bound for $\gamma$.

  Conversely, given $Q$ in $\F[z]/\langle\Phi_{mm'}(z)\rangle$, we obtain
  its preimage by replacing powers of $z$ by powers of $xx'$, reducing all
  exponents in $x$ modulo $m$, and all exponents in $x'$ modulo $m'$.  We
  then reduce the result modulo both $\Phi_m(x)$ and $\Phi_{m'}(x')$.  By
  the same argument as above, the cost is softly linear in $\varphi(mm')$.
\qed

\noindent{\bf Extension to several cyclotomic rings.}  The natural
generalization of the algorithm above starts with pairwise distinct
primes $\boldsymbol{p}=(p_1,\dots,p_t)$, non-negative exponent
$\boldsymbol{c}=(c_1,\dots,c_t)$ and variables
$\boldsymbol{x}=(x_1,\dots,x_t)$ over $\F$. Now, we define
$$\H:=\F[x_1,\dots,x_t]/\langle
\Phi_{{p_1}^{c_1}}(x_1),\dots,\Phi_{{p_t}^{c_t}}(x_t)\rangle;$$ when
needed, we will write $\H$ as
$\H_{\boldsymbol{p},\boldsymbol{c},\boldsymbol{x}}$. Finally, we let
$\mu:={p_1}^{c_1}\cdots {p_t}^{c_t}$; then, the dimension $\dim(\H)$ is
$\varphi(\mu)$.

\begin{lemma}\label{lemma:distinctP}
 There exist an $\F$-algebra isomorphism $\Gamma: \H \to
 \F[z]/\langle\Phi_{\mu}(z)\rangle$ given by $x_1 \cdots x_t \mapsto
 z$.  One can apply $\Gamma$ and its inverse to any input using
 $\tilde{O}(\dim(\H))$ operations in $\F$.
\end{lemma}
\noindent {\sc Proof.}
  We proceed iteratively. First, note that the cyclotomic polynomials
  $\Phi_{{p_i}^{c_i}}$ can all be computed in time $O(\varphi(\mu))$. 
  The isomorphism
  $\gamma: \F[x_1,x_2]/\langle \Phi_{{p_1}^{c_1}}(x_1),
  \Phi_{{p_2}^{c_2}}(x_2)\rangle \to \F[z]/\langle
  \Phi_{{p_1}^{c_1}{p_2}^{c_2}}(z)\rangle$
given in the previous paragraph extends coordinate-wise to an
  isomorphism
  $$\Gamma_1: \H \to \F[z,x_3,\dots,x_t]/\langle
  \Phi_{{p_1}^{c_1}{p_2}^{c_2}}(z),\Phi_{{p_3}^{c_3}}(x_3),\dots,\Phi_{{p_t}^{c_t}}(x_t)\rangle.$$
  By the previous lemma, $\Gamma_1$ and its inverse can be applied to
  any input in time $\tilde{O}(\varphi(\mu))$. Iterate this process
  another $t-2$ times, to obtain $\Gamma$ as a product
  $\Gamma_{t-1} \circ \cdots \circ \Gamma_1$. Since $t$ is logarithmic 
  in $\varphi(\mu)$, the proof is complete.
\qed

\noindent{\bf Tensor product of two prime-power cyclotomic rings, same
  $p$.}
We now consider cyclotomic polynomials of prime power
orders for a common prime $p$. As above, we start with two such polynomials.
Let thus $p$ be a prime. The key to the following algorithms is the
lemma below.  Let $c,c'$ be positive integers, with $c \ge
c'$, and let $x,y$ be indeterminates over $\F$. Define
\begin{align}
\mathbbm{a}&:=\F[x]/\Phi_{p^c}(x),  \\
\mathbbm{b}&:=\F[x,y]/\langle \Phi_{p^c}(x), \Phi_{p^{c'}}(y)\rangle = \mathbbm{a}[y]/\Phi_{p^{c'}}(y).
\end{align}
Note $\mathbbm{a}$ and $\mathbbm{b}$ have respective dimensions
$\varphi(p^c)$ and $\varphi(p^c) \varphi(p^{c'})$.

\smallskipback
\begin{lemma}
  There is an $\F$-algebra isomorphism $\theta: \mathbbm{b} \to
  \mathbbm{a}^{\varphi(p^{c'})}$ such that one can apply $\theta$ or
  its inverse to any inputs using $\tilde{O}(\dim(\mathbbm{b}))$ operations in $\F$.
\end{lemma}

\smallskipback
\noindent{\sc Proof.}
  Let $\xbar$ be the residue class of
  $x$ in $\A$. Then, in $\mathbbm{a}[y]$, $\Phi_{p^{c'}}(y)$ factors as
  $$\Phi_{p^{c'}}(y) =\prod_{\substack{1 \le i\le p^{c'}-1\\ \gcd(i,p)
      =1}} (y-\rho_i),$$ with $\rho_i:={\xbar}^{i p^{c-c'}}$ for all
  $i$.  Even though $\mathbbm{a}$ may not be a field, the Chinese
  Remainder theorem implies that $\mathbbm{b}$ is isomorphic to
  $\mathbbm{a}^{\varphi(p^{c'})}$; the isomorphism is given by
  $$\begin{array}{cccc}
    \theta: & \mathbbm{b} & \to & \mathbbm{a} \times \cdots \times \mathbbm{a}, \\
    & P & \mapsto& (P(\xbar,\rho_1),\dots,P(\xbar,\rho_{\varphi(p^{c'})}).
  \end{array}$$
  Arithmetic operations $(+,-,\times)$ in
  $\mathbbm{a}$ can be done in $\tilde{O}(\varphi(p^c))$ operations
  in $\F$. Starting from $\rho_1 \in \mathbbm{a}$, all other roots
  $\rho_i$ can then be computed in $O(\varphi(p^{c'}))$ operations in
  $\mathbbm{a}$ or $\tilde{O}(\dim(\mathbbm{b}))$
  operations in~$\F$. 
  
Applying $\theta$ and its inverse is done by means of fast evaluation
and interpolation~\cite[Chapter~10]{vzGathen13} in $\tilde{O}(\varphi(p^{c'}))$
operations in $\mathbbm{a}$, that is, $\tilde{O}(\deg(\mathbbm{b}))$ operations in $\F$
(the algorithms do not require that $\mathbbm{a}$ be a field).
\qed

\smallskip\noindent{\bf Extension to several cyclotomic rings.}
Let $p$ be as before, and consider now non-negative integers
$\boldsymbol{c}=(c_1,\dots,c_t)$ and variables $\boldsymbol{x}=(x_1,\dots,x_t)$. We
define the $\F$-algebra
$$\A:=\F[x_1,\dots,x_t]/\langle \Phi_{p^{c_1}}(x_1), \dots,
\Phi_{p^{c_t}}(x_t)\rangle,$$ which we will sometimes write
$\A_{p,\boldsymbol{c},\boldsymbol{x}}$ to make the dependency on $p$
and the $c_i$'s clear. Up to reordering the $c_i$'s, we can assume
that $c_1 \ge c_i$ holds for all $i$, and define as before
$\mathbbm{a}:=\F[x_1]/\Phi_{p^{c_1}}(x_1)$.

\begin{lemma}\label{lemma:A}
  There exists an $\F$-algebra isomorphism $\Theta: \A \to
  \mathbbm{a}^{\dim(\A)/\dim(\mathbbm{a})}$. This isomorphism and its
  inverse can be applied to any inputs using $\tilde{O}(\dim(\A))$
  operations in $\F$.
\end{lemma}
\noindent{\sc Proof.}
Without loss of generality, we can assume that all $c_i$'s are non-zero
(since for $c_i=0$, $\Phi_{p^{c_i}}(x_i)=x_i-1$,
so $\F[x_i]/\langle \Phi_{p^{c_i}}(x_i) \rangle = \F$).
We proceed iteratively. First, rewrite $\A$ as
$$\A=\mathbbm{a}[x_2,x_3,\dots,x_t]/\langle \Phi_{p^{c_2}}(x_2), \Phi_{p^{c_3}}(x_3), \dots,
\Phi_{{p_t}^{c_t}}(x_t)\rangle.$$ 
The isomorphism 
$\theta: \mathbbm{a}[x_2]/\Phi_{p^{c_2}}(x_2) \to \mathbbm{a}^{\varphi(p^{c_2})}$
introduced in the previous paragraph extends coordinate-wise
to an isomorphism 
$$\Theta_1: \A \to (\mathbbm{a}[x_3,\dots,x_t]/\langle
\Phi_{p^{c_3}}(x_3), \dots,
\Phi_{p^{c_t}}(x_t)\rangle)^{\varphi(p^{c_2})};$$ $\Theta_1$ and its
inverse can be evaluated in quasi-linear time $\tilde{O}(\dim(\A))$.
We now work in all copies of $\mathbbm{a}[x_3,\dots,x_t]/\langle
\Phi_{p^{c_3}}(x_3), \dots, \Phi_{p^{c_t}}(x_t)\rangle$ independently,
and apply the procedure above to each of them. Altogether we have
$t-1$ such steps to perform, giving us an isomorphism
$$\Theta = \Theta_{t-1} \circ \cdots \circ \Theta_1:
\A \to
\mathbbm{a}^{\varphi(p^{c_2}) \cdots \varphi(p^{c_t})}.$$
The exponent can be rewritten as $ \dim(\A)/\dim(\mathbbm{a})$, as claimed.
All $\Theta_i$'s and their inverses can be computed
in time $\tilde{O}(\dim(\A))$, and we do $t-1$ of them,
where $t$ is $O(\log(\dim(\A)))$. 
\qed

\noindent{\bf Decomposing certain $p$-group algebras.}  The prime $p$
and indeterminates $\boldsymbol{x}=(x_1,\dots,x_t)$ are as before; we now consider
positive integers $\boldsymbol{b}=(b_1,\dots,b_t)$, and the $\F$-algebra
\[
\begin{array}{ccl}
\B&:=&\F[x_1,\dots,x_t]/\langle x_1^{p^{b_1}}-1,\dots,x_t^{p^{b_t}}-1\rangle\\$$
&=& \F[x_1]/\langle x_1^{p^{b_1}}-1 \rangle \otimes \cdots \otimes \F[x_t]/\langle x_t^{p^{b_t}}-1 \rangle.
\end{array}
\]
If needed, we will write $\B_{p,\boldsymbol{b},\boldsymbol{x}}$ to make the dependency
on $p$ and the $b_i$'s clear. This is the $\F$-group algebra
of $\Z/p^{b_1}\Z \times \cdots \times \Z/p^{b_t}\Z$.

\begin{lemma}\label{lemma:alg}
  There exists a positive integer $N$, non-negative integers
  $\boldsymbol{c}=(c_1,\dots,c_N)$ and  an
  $\F$-algebra isomorphism 
  $$\Lambda: \B \to \D= \F[z]/\langle \Phi_{p^{c_1}}(z) \rangle \times \cdots \times \F[z]/\langle \Phi_{p^{c_N}}(z)\rangle.$$
  One can apply the isomorphism and its inverse to any 
  input using $\tilde{O}(\dim(\B))$ operations in $\F$.
\end{lemma}
\noindent{\sc Proof.}
For $i \le t$, we have the factorization
$$x_i^{p^{b_i}}-1 = \Phi_1(x_i) \Phi_p(x_i) \Phi_{p^2}(x_i) \cdots
\Phi_{p^{b_i}}(x_i);$$ note that $\Phi_1(x_i)=x_i-1$.  The factors may
not be irreducible, but are pairwise coprime, so we have a
Chinese Remainder isomorphism
\[
  \lambda_i: \F[x_i]/\langle x_i^{p^{b_i}}-1 \rangle \to \F[x_i]/\langle \Phi_1(x_i)\rangle
  \times \cdots \times  \F[x_i]/\langle \Phi_{p^{b_i}}(x_i)\rangle.
\]
It and its inverse can be computed  
in $\tilde{O}(p^{b_i})$ operations in $\F$~\cite[Chapter~10]{vzGathen13}. 
This gives an $\F$-algebra isomorphism
$$\lambda: \B \to \prod_{c_1=0}^{b_1} \cdots \prod_{c_t=0}^{b_t} \A_{p,\boldsymbol{c},\boldsymbol{x}},$$
with $\boldsymbol{c}=(c_1,\dots,c_t)$. Together with its inverse, 
$\lambda$ can be computed in $\tilde{O}(\dim(\B))$ operations in $\F$.
Composing with the result in Lemma~\ref{lemma:A}, this gives
us an isomorphism
$$\Lambda: \B \to \D:=\prod_{c_1=0}^{b_1} \cdots \prod_{c_t=0}^{b_t}
\mathbbm{a}_{\boldsymbol{c}}^{D_{\boldsymbol{c}}},$$ where
$\mathbbm{a}_{\boldsymbol{c}} = \F[z]/\langle \Phi_{p^c}(z)\rangle$,
with $c =\max(c_1,\dots,c_t)$ and $D_{\boldsymbol{c}} =
\dim(\A_{t,\boldsymbol{c},\boldsymbol{x}})/\dim(\mathbbm{a}_{\boldsymbol{c}})$. As
before, $\Lambda$ and its inverse can be computed in quasi-linear time
$\tilde{O}(\dim(\B))$. \qed

As for $\B$, we will write $\D_{p,\boldsymbol{b},\boldsymbol{x}}$ if needed; it is
well-defined, up to the order of the factors.

\smallskip

\noindent{\bf Main result.} Let $G$ be an abelian group.  We can
write the elementary divisor decomposition of $G$ as $G = G_1 \times
\cdots \times G_s$, where each $G_i$ is of prime power order
$p_i^{a_i}$, for pairwise distinct primes $p_1,\dots,p_s$, so that
$|G| = p_1^{a_1} \cdots p_s^{a_s}$. Each $G_i$ can itself be written
as a product of cyclic groups, $G_i = G_{i,1} \times \cdots \times
G_{i,t_i}$, where the factor $G_{i,j}$ is cyclic of order
${p_i}^{b_{i,j}}$, with $b_{i,1} \le \cdots \le b_{i,t_i}$ and $b_{i,1} + \cdots +
b_{i,t_i} = a_i$. We henceforth assume that generators
$\gamma_{1,1},\dots,\gamma_{s,t_s}$ of respectively
$G_{1,1},\dots,G_{s,t_s}$ are known, and that elements of $\F[G]$ are
given on the power basis in $\gamma_{1,1},\dots,\gamma_{s,t_s}$. 

%% Were
%% this not the case, given arbitrary generators $g_1,\dots,g_r$ of $G$, with
%% orders $e_1,\dots,e_r$, a brute-force solution would factor each $e_i$
%% (factoring $e_i$ takes $o(e_i)$ bit operations on a standard RAM), so
%% as to write $\langle g_i \rangle$ as a product of cyclic groups of
%% prime power orders, from which the required decomposition follows.

\begin{proposition}
  Given $\beta \in \F[G]$, written in the basis
  $\gamma_{1,1},\dots,\gamma_{s,t_s}$, one can test if $\beta$ is a
  unit in $\F[G]$ in time $\tilde{O}(|G|)$.
\end{proposition}
From the factorization $G = G_1 \times \cdots \times G_s$, we deduce
that the group algebra $\F[G]$ is the tensor product $\F[G_1]
\otimes \cdots \otimes \F[G_s]$. Furthermore, the 
factorization $G_i = G_{i,1} \times \cdots \times G_{i,t_i}$
implies that $\F[G_i]$ is isomorphic, as an $\F$-algebra, to
$$\F[x_{i,1},\dots,x_{i,t_i}]/\left \langle
x_{i,1}^{p_i^{b_{1}}}-1,\dots,x_{i,t_i}^{p_i^{b_{i,t_i}}}-1\right\rangle
=\B_{p_i,\boldsymbol{b}_i,\boldsymbol{x}_i},$$ with $\boldsymbol{b}_i
= (b_{i,1},\dots,b_{i,t_i})$ and $\boldsymbol{x}_i =
(x_{i,1},\dots,x_{i,t_i})$. Given $\beta$ on the power basis in
$\gamma_{1,1},\dots,\gamma_{s,t_s}$, we obtain its image $B$ in
$\B_{p_1,\boldsymbol{b}_1,\boldsymbol{x}_1} \otimes \cdots \otimes
\B_{p_s,\boldsymbol{b}_s,\boldsymbol{x}_s}$ simply by renaming
$\gamma_{i,j}$ as $x_{i,j}$, for all $i,j$.

For $i \le s$, by Lemma~\ref{lemma:alg}, there exist integers
$c_{i,1},\dots,c_{i,N_i}$ such that
$\B_{p_i,\boldsymbol{b}_i,\boldsymbol{x}_i}$ is isomorphic to an
algebra $\D_{p_i, \boldsymbol{b}_i, z_i}$, with factors 
$\F[z_i]/\langle \Phi_{{p_i}^{c_{i,j}}}(z_i) \rangle$.
By distributivity of the tensor product over direct products, we
deduce that $\B_{p_1,\boldsymbol{b}_1,\boldsymbol{x}_1} \otimes \cdots
\otimes \B_{p_s,\boldsymbol{b}_s,\boldsymbol{x}_s}$ is isomorphic to
the product of algebras
 \begin{equation}\label{eq:prod}
\text{\small $\prod$}_{\boldsymbol{j}}~ \F[z_1,\dots,z_s]/
\langle \Phi_{{p_1}^{c_{1,j_1}}}(z_1),\dots, \Phi_{{p_s}^{c_{s,j_s}}}(z_s) \rangle,   
 \end{equation}
for all indices $\boldsymbol{j}=(j_1,\dots,j_s)$, with
$j_1 =1,\dots,N_1,\dots,j_s=1,\dots,N_s$;
call $\Gamma$ the isomorphism. Given $B$ in $\B_{p_1,\boldsymbol{b}_1,\boldsymbol{x}_1} \otimes
\cdots \otimes \B_{p_s,\boldsymbol{b}_s,\boldsymbol{x}_s}$,
Lemma~\ref{lemma:alg} also implies that $B':=\Gamma(B)$ can be
computed in time $\tilde{O}(|G|)$ (apply the isomorphism
corresponding to $\boldsymbol{x}_1$ coordinate-wise with respect to
all other variables, then deal with $\boldsymbol{x}_2$, etc).
The codomain in~\eqref{eq:prod} is the product of all $\H_{\boldsymbol{p},\boldsymbol{c}_{\boldsymbol{j}},\boldsymbol{z}}$,
with 
$$\boldsymbol{p}=(p_1,\dots,p_s),\quad \boldsymbol{c}=(c_{1,j_1},\dots,c_{s,j_s}),\quad \boldsymbol{z}=(z_1,\dots,z_s).$$
Apply Lemma~\ref{lemma:distinctP} to all 
$\H_{\boldsymbol{p},\boldsymbol{c}_{\boldsymbol{j}},\boldsymbol{z}}$ to obtain
an $\F$-algebra isomorphism
$$\Gamma': \text{\small $\prod$}_{\boldsymbol{j}}~
\H_{\boldsymbol{p},\boldsymbol{c}_{\boldsymbol{j}},\boldsymbol{z}} \to
\text{\small $\prod$}_{\boldsymbol{j}} ~\F[z]/\langle
\Phi_{d_{\boldsymbol{j}}}(z) \rangle,$$ for certain integers
$d_{\boldsymbol{j}}$. The lemma implies that given $B'$,
$B'':=\Gamma'(B')$ can be computed in softly linear time
$\tilde{O}(|G|)$ as well. Invertibility of $\beta \in \F[G]$ is
equivalent to $A''$ being invertible, that is, to all its components
being invertible in the respective factors $\F[z]/\langle
\Phi_{d_{\boldsymbol{j}}}(z) \rangle$. Invertibility in such an
algebra can be tested in softly linear time by applying the fast
extended GCD algorithm~\cite[Chapter~11]{vzGathen13}, so our conclusion follows. 
With Proposition \ref{prop:abelian}, this proves
the first part of Theorem \ref{thm:main}.

%%%%%%%%%%%%%%%%%%%%%%%%%%%%%%%%%%%%%%%%%%%%%%%%%%%%%%%%%%%%

\subsection{Metacyclic Groups}

We next study the invertibility problem for a
metacyclic group $G$. We use an injective homomorphism, whose image will
be easy to compute. This is the object of the following lemma, where
the map is inspired by the one used in \cite[\S 47]{Curtis}.

Assume that $G = \langle \sigma , \tau : \sigma^m = 1, \tau^s =
\sigma^t, \tau^{-1} \sigma \tau = \sigma^r \rangle$, where $r^s = 1
\bmod m$ and $rt = t \bmod m$; in particular, $n=|G|$ is equal to
$ms$. Define $\A:=\F[z]/\langle z^m-1\rangle$ and let $\zbar$ be the
image of $z$ in $\A$.

\begin{lemma}\label{prop:metinjection}
The mapping $\psi: \F [G] \to M_s(\A)$ where
\[
\sigma \mapsto \mathrm{Diag}(\zbar, \zbar^r , \ldots, \zbar^{r^{s-1}})
,\; \tau \mapsto 
\left[ \begin{array}{l|l}
0 & \zbar\\
\hline
\mat I_{s-1} & 0
\end{array}
\right]
\]
is an injective $\F$-algebra homomorphism.
\end{lemma}
\noindent{\sc Proof.}
It is straightforward to verify that $\psi(\sigma)^m = \mat I_m$,
$\psi(\tau)^s = \psi(\sigma)^t$ and $\psi_i(\sigma) \psi_i(\tau)
=\psi(\tau) \psi_i(\sigma)^r$; this shows that $\psi$ is a well-defined $\F$-algebras homomorphism.

Take $\beta \in \F[G]$, and write it $\beta = \sum_{j = 0}^{s-1}
\left( \sum_{i = 0}^{m-1} b_{i,j} \sigma^i \right) \tau^j$. For
$j=0,\dots,s-1$, define $F_j(x) := \sum_{i = 0}^{m-1} b_{i,j}x^i \in
\F[x]$ and, for $1 \leq i,j \leq s$,
$F_{i,j} := F_{i-1}(\zbar^{r^{j-1}}).$
Then, $\psi(\beta)$ is the matrix
\begin{equation}\label{eq:injection}
\left[\begin{array}{llll}
F_{1,1} &  \cdots	&	\zbar F_{3,s-1} & \zbar F_{2,1}\\
F_{2,2} & F_{1,2}& \cdots & \zbar F_{3,s}\\
\vdots &\ddots & \ddots& \vdots\\
F_{s,s}& \cdots & F_{2,s}	& F_{1,s}
\end{array}
\right].
\end{equation}
If $\beta$ is in $\mathrm{Ker}(\psi)$, we get $F_i(\zbar) =0$, that is,
$F_i \bmod (z^m-1)=0$, for $0 \leq i < s$.  All $F_i$'s have degree 
less than $m$, so they are all zero. \qed


We finally give two algorithms that test whether $\psi(\beta) \in
M_s(\A)$ is invertible, for a given $\beta$ in $\F[G]$. Minor
difficulties will arise as we work over $\A$, since $\A$ is not a
field, but a product of fields (if the irreducible factorization of
$z^m-1$ in $\F[z]$ is known, we can use the Chinese Remainder theorem
and work in field extensions of $\F$).

\begin{corollary}\label{coro:test_meta}
  Given $\beta$ in $\F[G]$, one can test if $\beta$ is a unit in
  $\F[G]$ either by a deterministic algorithm that uses
  $\tilde{O}(s^{2.7} m)$ operations in $\F$, or a Monte Carlo one that
  uses $\tilde{O}(n^2)$ operations in $\F$.
\end{corollary}
The second statement provides the last part of the proof of Theorem
\ref{thm:main}. Note that the first algorithm gives a better cost in
many cases. For instance, if $s \leq m$, the first algorithm uses
$O(n^{1.85})$ operations in $\F$. This happens if $s$ is prime, since
then the number ${(m- \gcd(m,r-1))}/{s}$ is a positive integer,
which implies $s \leq m$ (see \cite[Theorem 47.12, Corollary 47.14
]{Curtis}).


\noindent{\sc First algorithm.}
  The first algorithm uses fast linear algebra algorithms over the ring
  $\A$. Here, we start from $\beta$ written as $\beta = \sum_{j =
    0}^{s-1} \left( \sum_{i = 0}^{m-1} b_{i,j} \sigma^i \right) \tau^j
  \in \F[G]$. Then, the proof of the previous lemma shows an explicit
  formula for $\psi(\beta)$. In order to compute this matrix, we note
  that $\zbar^{r^{j-1}} = \zbar^{r^{j-1} \bmod m}$; computing this
  element and its powers requires no arithmetic operation, so that the
  coefficients of each $F_{i,j}$ are obtained in linear time $O(m)$.
  Hence the matrix $\psi(\beta)$ can be computed in time $O(s^2m)$.

  Next, we have to determine whether $\psi(\beta)$ is a unit (the
  injectivity of $\psi$ implies that this is the case if and only if
  $\beta$ itself is a unit). This amounts to computing the determinant
  of this matrix, which can be done in $\tilde{O}(s^{2.7} m)$
  operations in $\F$, using the determinant algorithm
  of~\cite[Section~6]{KaVi04}. \qed


%% Before giving our second algorithm, let us point out that matrix-vector
%% products by $\psi(\beta)$ can be done fast.

\begin{lemma}\label{lem:multpsi}
  Given $\beta$ in $\F[G]$ and $\boldsymbol{v}$ in $\A^s$, one can
  compute $\psi(\beta) \boldsymbol{v} \in \A^s$ using $\tilde{O}(s
  m^2)$ operations in $\F$.
\end{lemma}
\noindent{\sc Proof.}
  We use the basis of  $\F[G]$ of~\eqref{pres2}, writing $\beta =
  \sum_{i = 0}^{m-1} \left( \sum_{j = 0}^{s-1}  b_{i,j} \tau^i
  \right) \sigma^j = \sum_{i
    = 0}^{m-1} B_i(\tau) \sigma^i$, for some $B_0,\dots,B_{m-1}$ in
  $\F[z]$ of degree less than $s$.

 Given $\boldsymbol{v}$ as above, we compute all $B_i(\psi(\tau))
 \psi(\sigma)^i \boldsymbol{v}$ independently, and add them to obtain
 $\psi(\beta) \boldsymbol{v}$. Hence, let us fix an index $i$ in
 $\{0,\dots,m-1\}$.
 The vector $\psi(\sigma)^i \boldsymbol{v}$ can be obtained by
 multiplying each entry of $\boldsymbol{v}$ by a power of $\zbar$;
 this takes $\tilde{O}(sm)$ operations in $\F$. Then, since
 $\psi(\tau)$ is the matrix of multiplication by $y$ in $\A[y]/\langle
 y^s-\zbar\rangle$, $B_i(\psi(\tau))$ is the matrix of multiplication
 by $B_i(y)$ in $\A[y]/\langle y^s-\zbar\rangle$. Thus, applying this
 matrix to a vector also takes time $\tilde{O}(sm)$.
 Adding a factor of $m$ to account for all indices $i$ gives
 the result.
\qed





\noindent{\sc Second algorithm for Corollary~\ref{coro:test_meta}.}
  The second algorithm uses \citeauthor{Wiedemann86}'s
  \citeyear{Wiedemann86} algorithm,
  and its extension by~\citeN{KaSa91}. Extra care will be needed to
  accommodate the fact that $\A$ has zero-divisors. Let
  $F_1,\dots,F_s$ be the (unknown) irreducible factors of $z^m-1$ in
  $\F[z]$ and define $\A_i := \F[z]/\langle F_i \rangle$ for
  $i=1,\dots,s$. We write $\pi_i: \A \to\A_i$ for the canonical
  projection, and extend the notation to matrices over $\A$.

  For $\beta$ in $\F[G]$, $\mat M:=\psi(\beta)$ is invertible if and only
  if all $\mat M_i := \pi_i(\mat M)$ are. We are going to use the algorithm
  of~\cite[Section~4]{KaSa91} to compute the rank of all these matrices
  (these ranks are well-defined, since all $\A_i$'s are fields).  Let
  $\mat L$ and $\mat U$ be respectively random lower triangular and upper
  triangular Toeplitz matrices over $\A$, and define
  $\mat M':=\mat L \mat M \mat U \in M_s(\A)$. Finally, let $\mat M''$ be
  $\mat M'$, to which we adjoin a bottom row and a rightmost column of
  zeros (so it has size $s+1$), let $\mat M''_i :=\pi_i(\mat M'')$ and let
  $r_i :=\text{rank}(\mat M''_i)$, $i=1,\dots,s$. Then, all $r_i$'s are
  less than $s+1$, and $\mat M$ is invertible if and only if $r_i=s$ for
  all $i$.

  The condition that $\mat M''_i$ has rank less than $s+1$ makes it
  possible to apply~\cite[Lemma~2]{KaSa91}: for generic
  $\boldsymbol{u}_i, \boldsymbol{v}_i$ in $\A_i^{s+1}$ and diagonal
  matrix $\mat X$ in $M_{s+1}(\A_i)$, the minimal polynomial of the
  sequence $(\boldsymbol{u}_i^T (\mat M''_i \mat X_i)^j
  \boldsymbol{v}_i)_{j \ge 0}$ has degree $r_i+1$.

  To compute these degrees without knowing the factorization $z^m-1 =
  F_1 \cdots F_s$, we choose random $\boldsymbol{u}, \boldsymbol{v}$
  in $\A^{s+1}$ and diagonal matrix $\mat X$ in $M_{s+1}(\A)$.  Then,
  we compute $2s$ terms in the sequence $(\gamma_j)_{j\ge 0}$, with
  $\gamma_j:=\boldsymbol{u}^T (\mat M'' \mat X)^j
  \boldsymbol{v}$. Since multiplication by $\mat L$, $\mat U$ and
  $\mat X$ all take quasi-linear time $\tilde{O}(sm)$,
  Lemma~\ref{lem:multpsi} shows that one product by $\mat M'' \mat X$
  takes $\tilde{O}(sm^2)$ operations in $\F$. Hence, all required
  terms can be obtained in $\tilde{O}(s^2m^2)=\tilde{O}(n^2)$
  operations in $\F$.

  Finally, we apply the fast Euclidean algorithm to $\sum_{j=0}^{2s-1}
  \gamma_j y^j$ and $y^{2s}$ in the ring $\A[y]$ to find the ranks
  $r_1,\dots,r_s$.  Since $\A$ is not a field, we rely on the
  algorithm of~\cite{AcCoMa03,DaMMMScXi06}. Using $\tilde{O}(sm)$
  operations in $\F$, it reveals a partial factorization of $z^m-1$ as
  $G_1 \cdots G_t$ (the factors may not be irreducible) and integers
  $\rho_j$, $j=1,\dots,t$, such that for all $i \le s$, $j\le t$, if
  $F_i$ divides $G_j$, then $r_i=\rho_j$. This allows us to determine all
  $r_i$'s, and thus decide whether $\psi(\beta)$ is singular.\qed






%%% Local Variables:
%%% mode: latex
%%% TeX-master: "NormalBasisCharZero"
%%% End:



\citestyle{acmauthoryear}
\bibliographystyle{ACM-Reference-Format}
\bibliography{NormalBasisCharZero} 

\end{document}

%%% Local Variables:
%%% mode: latex
%%% TeX-master: t
%%% End:
