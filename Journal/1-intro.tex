\section{Introduction}

For a finite Galois extension field $\K/\F$, with Galois group $G =
\mathrm{Gal}(\K/\F)$, an element $\alpha \in \K$ is called
\emph{normal} if the set of its Galois conjugates $G \cdot \alpha = \{
g(\alpha): g\in G\}$ forms a basis for $\K$ as a vector space over
$\F$. The existence of normal element for any finite Galois extension
is classical, and constructive proofs are provided in most algebra
texts (see, e.g., \citep[Section 6.13]{Lang}).
%\cite{Wae70}, Section~8.11).
 
While there is a wide range of well-known applications of normal bases in
finite fields, such as fast exponentiation~\citep{GaGaPaSh00}, there also
exist applications of normal elements in characteristic zero.  For instance,
in multiplicative invariant theory, for a given permutation lattice and
related Galois extension, a normal basis is useful in computing the
multiplicative invariants explicitly~\citep{Jam18}.

A number of algorithms are available for finding a normal element in
characteristic zero fields and finite fields.  Because of their immediate
applications in finite fields, algorithms for determining normal elements
in this case are most commonly seen.  A fast randomized algorithm for
determining a normal element in a finite field $\FF_{q^n}/\FF_q$, where
$\FF_{q^n}$ is the finite field with $q^n$ elements for any prime power $q$
and integer $n>1$, is presented by \citeN{GatGie90}, with a cost of
$O(n^2+n\log q)$ operations in $\FF_q$.  A faster randomized algorithm is
introduced by \citeN{KalSho98}, with a cost of $O(n^{1.815}\log q)$
operations in $\FF_q$.  In the bit complexity model, Kedlaya and Umans showed
how to reduce the exponent of $n$ to $1.5+\varepsilon$ (for any
$\varepsilon > 0$), by leveraging their quasi-linear time algorithm for
{\em modular composition}~\citep{KeUm11}. \cite{LenstraNormal} introduced a
deterministic algorithm to construct a normal element which uses $n^{O(1)}$
operations in $\FF_{q^n}/\FF_q$.  To the best of our knowledge, the
algorithm of \cite{AugCam94} is the most efficient deterministic method,
with a cost of $O(n^3+n^2\log q)$ operations in $\FF_q$.

In characteristic zero, \cite{SchSte93} gave an algorithm for finding
a normal basis of a number field over $\QQ$ with a cyclic Galois group
of cardinality $n$ which requires $n^{O(1)}$ operations in $\QQ$.
\cite{Pol94} gives an algorithm for the more general case of finding a
normal basis in an abelian extension $\K/\F$ which requires $n^{O(1)}$
operations in $\F$.  More generally in characteristic zero, for any
Galois extension $\K/\F$ of degree $n$ with Galois group given by a
collection of $n$ matrices, \cite{Girstmair} gives an algorithm which
requires $O(n^4)$ operations in $\F$ to construct a normal element in
$\K$.

In this paper we present a new randomized algorithm that decides
whether a given element in either an abelian or a metacyclic extension
is normal, with a runtime \alg in the degree $n$ of the extension. The
costs of all algorithms are measured by counting \emph{arithmetic
  operations} in $\F$ at unit cost.  Questions related to the
bit-complexity of our algorithms are challenging, and beyond the scope
of this paper.

Our main conventions are the following.
\begin{assumption}
  \label{assum}
  Let $\K/\F$ be a finite Galois extension presented as
  $\K=\F[x]/\langle P(x)\rangle$, for an irreducible polynomial $P\in
  \F[x]$ of degree $n$, with $\F$ of characteristic zero. Then,
  \begin{itemize}
  \item elements of $\K$ are written on the power basis $1,\xbar,\dots,\xbar^{n-1}$,
    where $\xbar := x \bmod P$;
  \item elements of $G$ are represented by their action on $\xbar$.
  \end{itemize}
\end{assumption}

In particular, for $g \in G$ given by means of $\gamma:=g(\xbar) \in \K$,
and $\beta = \sum_{0\leq i<n}\beta_i\xbar^i\in\K$, the fact that $g$ is an
$\F$-automorphism implies that $g(\beta)$ is equal to $\beta(\gamma)$, the
polynomial composition of $\beta$ at $\gamma$ (reduced modulo $P$).

Our algorithms combine techniques and ideas of~\cite{GatGie90} and
\cite{KalSho98}: $\alpha \in \K$ is normal if and only if the element
$S_\alpha := \sum_{g \in G} g(\alpha)g \in \K[G]$ is invertible in the
group algebra $\K[G]$.  However, writing down $S_\alpha$ involves
$\Theta(n^2)$ elements in $\F$, which precludes a subquadratic
runtime. Instead, knowing $\alpha$, the algorithms use a randomized
reduction to a similar question in $\F[G]$, that amounts to applying a
random projection $\ell:\K\to\F$ to all entries of $S_\alpha$, giving
us an element $s_{\alpha,\ell} \in \F[G]$. For that, we adapt
algorithms from~\citep{KalSho98} that were developed for Galois groups
of finite fields.

Having $s_{\alpha,\ell}$ in hand, we need to test its
invertibility. In order to do so, we present an algorithm in the
abelian case which relies on the fact that $\F[G]$ is isomorphic to a
multivariate quotient polynomial ring by an ideal $(x^{e_i}_i-1)_{1
  \leq i \leq m}$, where $e_i$'s are positive integers. For metacyclic
groups, we exploit the block-Hankel structure of the matrix of
multiplication by $s_{\alpha,\ell}$. Remark that these questions on
the cost of arithmetic operations in $\F[G]$ are closely related to
that of Fourier transform over $G$, and it is worth mentioning that
there is a vast literature on fast algorithms for Fourier transforms
(over the base field $\C$). Relevant to our current context, consider
\citep{ClaMu04} and \citep{MaRockWol18} and references therein for
details. At this stage, it is not clear how we can apply these methods
in our context (where we work over an arbitrary $\F$, not necessarily 
algebraically closed).

This paper is written from the point of view of obtaining improved
asymptotic complexity estimates. Since our main goal is to highlight
the exponent (in $n$) in our runtime analyses, costs are given using
the soft-O notation: $S(n)$ is in $\tilde{O}(T(n))$ if it is in
$O(T(n) \log(T(n))^c)$, for some constant $c$.

The first main result of this paper is the following theorem; we use a
constant $\omega(4/3)$ that describes the cost of certain rectangular
matrix products (see the end of this section).
\begin{theorem}\label{thm:main}
  Under Assumption \ref{assum}, one can test whether $\alpha \in \K$
  is normal using $\thecost$ operations in $\F$ if $G$ is either
  abelian or metacyclic, with $(3/4)\cdot\omega(4/3)<1.99$. The
  algorithms are randomized.
\end{theorem}
% Although the cost is quadratic in the size of input for a general
% metacyclic group, in many cases it will be (slightly) subquadratic,
% under specific conditions on the parameters defining $G$ (see
% Section~\ref{sec:invertibility}).

Once $\alpha$ is known to be normal, we also discuss the cost 
of conversion between the power basis $1,\xbar,\dots,\xbar^{n-1}$
of $\K$ and its normal basis $G\cdot \alpha$. Again inspired by
previous work of~\cite{KalSho98}, we obtain the following results.
\begin{theorem}\label{thm:main2}
  Under Assumption \ref{assum}, if $G$ is either abelian or metacyclic
  and $\alpha \in \K$ is known to be normal, we can perform basis
  conversion between the power basis $1,\xbar,\dots,\xbar^{n-1}$ of
  $\K$ and its normal basis $G\cdot \alpha$ using $\thecost$
  operations in $\F$. The algorithms are randomized.
\end{theorem}
In both theorems, the runtime is barely subquadratic, and the exponent
$1.99$ is obtained through fast matrix multiplication algorithms that
are most likely impractical for reasonable $n$. However, these results
show in particular that we can perform basis conversions without
writing down the normal basis itself (which would require
$\Theta(n^2)$ elements in $\K$).

Section \ref{sec:pre} of this paper is devoted to definitions and
preliminary discussions.  In Section \ref{sec:osum}, two
subquadratic-time algorithms are presented for the randomized reduction
of our main question to invertibility testing in $\F[G]$, for
respectively abelian and metacyclic groups.  In Section
\ref{sec:invertibility}, we show that the latter problem can be solved
in quasi-linear time for an abelian group; for metacyclic
groups, we give a subquadratic time algorithm based on structured linear 
algebra algorithms. Finally, Section~\ref{sec:conversion} proves
Theorem~\ref{thm:main2}.

Our algorithms make extensive of known algorithms for polynomial and
matrix arithmetic; in particular, we use repeatedly the fact that
polynomials of degree $d$ in $\F[x]$, for any field $\F$ of
characteristic zero, can be multiplied in $\tilde{O}(n)$ operations in
$\F$~\citep{ScSt71}. As a result, arithmetic operations
$(+,\times,\div)$ in $\K$ can all be done using $\tilde{O}(n)$
operations in $\F$~\citep{vzGathen13}.

For matrix arithmetic, we will rely on some non-trivial results on
rectangular matrix multiplication initiated by \cite{LoRo83}. For $k \in
\mathbb{R}$, we denote by $\omega(k)$ a constant such that over any
ring, matrices of sizes $(n,n)$ by $(n,\lceil n^k \rceil)$ can be
multiplied in $O(n^{\omega(k)})$ ring operations (so $\omega(1)$ is
the usual exponent of square matrix multiplication, which we simply
write $\omega$).  The sharpest values known to date for most
rectangular formats are by~\cite{LeGall}; for $k=1$, the best known
value is $\omega \le 2.373$ by \citeN{LeGall14}. Over a field, we will
frequently use the fact that further matrix operations (determinant or
inverse) can be done in $O(n^\omega)$ base field operations.


Part of the results of this paper (Theorem~\ref{thm:main} for abelian
groups) were already published in the conference
paper~\citep{GiJaSc19}.

%%% Local Variables:
%%% mode: latex
%%% TeX-master: "NormalBasisCharZero"
%%% End:
