\section{Basis Conversion}\label{sec:conversion}

We conclude this paper with algorithms for basis conversion: assuming
we know that $\alpha$ is normal, we show how to perform the
change-of-basis between the power basis of $\K/\F$ and the normal
basis $G\cdot \alpha$. The techniques used below are inspired by 
those used by~\cite[Section~4]{KalSho98} in the case of extensions
of finite fields.

\subsection{From normal to power basis} Suppose $G = \lbrace g_1, \ldots, g_n
\rbrace$, $\alpha$ is a normal element of $\K/\F$ and we are given $u\in \K$ 
as $u=\sum_{i=1}^n u_i g_i(\alpha)$. In order to write
$u $ in the power basis, we have to compute the matrix-vector 
product
\begin{equation}\label{eq:norm2pwrmat}
\left[\begin{array}{ccc}
\boldsymbol \gamma_1 & \cdots & \boldsymbol \gamma_n 
\end{array}\right]\cdot 
\begin{bmatrix}
u_1 \\ \vdots \\ u_n
\end{bmatrix},
\end{equation}
where for $i=1,\dots,n$, $\boldsymbol \gamma_i \in \F^{n\times 1}$ is
the coefficient vector of $g_i(\alpha)$. As already pointed out by
Kaltofen and Shoup, this shows that conversion from normal to power
basis is the transpose problem of computing the ``projected'' orbit
sum $s_{\alpha,\ell}$, which we solved in Section \ref{sec:osum}. 

The transposition principle then allows us to derive runtime estimates
for the conversion problem; below, we present an explicit procedure
derived from the algorithm in Subsection~\ref{ssec:proj_abelian}.  As
in that section, we give the algorithm in the general case of a
polycyclic group $G$ presented as
\[
G =\{ g_r^{i_r} \cdots g_1^{i_1}, \text{~with~}  0 \leq i_j < e_j \text{~for~} 1 \leq j \leq r\}.
\]
With indices $i_1,\dots,i_r$ as above, we are given a family of 
coefficients $u_{i_1,\dots,i_r}$ in $F$, and we expand the sum
$u=\sum_{i_1,\dots,i_r} u_{i_1,\dots,i_r} g_r^{i_r} \cdots g_1^{i_1}(\alpha)$
on the power basis of $\K/\F$. For this, we let $z \in\{1,\dots,r\}$ be the index
defined in Subsection~\ref{ssec:proj_abelian}.

\smallskip\noindent \textbf{Step 1.} Apply Lemma \ref{lem:selfcomp},
with $\alpha_{i_1,\dots,i_r} = \alpha$ for all $i_1,\dots,i_r$, to get
$$ G_{i_z,\dots,i_1}=g_z^{i_z} \cdots
g_1^{i_1}(\alpha),$$ for all indices $i_1,\dots,i_z$ such that $0\leq
i_m < e_m$ holds for $m=1,\dots,z-1$ and $0\leq i_z <s_z=
\lceil{\sqrt{n}}/(e_1 \cdots e_{z-1})\rceil$.  As in
Subsection~\ref{ssec:proj_abelian}, the cost of this step is
$\osumcosttilde$.

\smallskip\noindent\textbf{Step 2.} Compute $G=g_z^{s_z}$, for $s_z$
as above. The cost is is negligible compared to the cost of the
previous step.

\smallskip\noindent\textbf{Step 3.} Compute the matrix product 
$\mat U \mat \Gamma$, where
\begin{itemize}
\item[$\bullet$] $\mat U$ is the matrix over $\F$ having $\lceil
  e_z/s_z\rceil e_{z+1} \cdots e_r$ rows and $e_1 \cdots e_{z-1} s_z$
  columns built as follows. Rows are indexed by $(j_z,\dots,j_r)$,
  with $0\le j_z < \lceil e_z/s_z\rceil$ and $0 \le j_m < e_m$ for all
  other indices; columns are indexed by $(i_1,\dots,i_z)$, with $0\le
  i_z < s_z$ and $0 \le i_m < e_m$ for all other indices; the entry at
  rows $(j_z,\dots,j_r)$ and column $(i_1,\dots,i_z)$ is
  $u_{i_1,\dots,i_z + s_z j_z, j_{z+1},\dots,j_r}$.
\item[$\bullet$] $\mat \Gamma$ is the matrix with $e_1 \cdots e_{z-1}
  s_z$ rows (indexed in the same way as the columns of $\mat U$) and $n$
  columns, whose row of index $(i_1,\dots,i_z)$ contains the
  coefficients of $ G_{i_z,\dots,i_1}$ (on the power basis of $\K$)
\end{itemize}
As established in Subsection~\ref{ssec:proj_abelian}, the row and column
dimensions of $\mat U$ are $O(\sqrt n)$, so this product can 
be computed in $O(n^{(1/2)\cdot\omega(2)})$ operations in $\F$. The rows of 
the resulting matrix give the coefficients of 
$$ H_{j_{z+1},\dots,j_r} = \sum_{i_1,\dots,i_z} u_{i_1,\dots,i_z+s_z j_z,\dots,j_r} g_z^{i_z} \cdots
g_1^{i_1}(\alpha),$$
for all indices $(j_z,\dots,j_r)$ and $(i_1,\dots,i_z)$ as above.

\smallskip\noindent\textbf{Step 4.} Compute and add all
$$g_r^{j_1} \cdots g_{z+1}^{j_{z+1}} G^{j_z} ( H_{j_{z+1},\dots,j_r}
),$$ for indices $(j_z,\dots,j_r)$ as above; their sum is precisely
the input element $u=\sum_{i_1,\dots,i_r} u_{i_1,\dots,i_r} g_r^{i_r}
\cdots g_1^{i_1}(\alpha)$, written on the power basis.

This is done by a second call to Lemma \ref{lem:selfcomp}, for the same
asymptotic cost as in Step 1. Summing all costs, we arrive
at an overall runtime of $\osumcosttilde$ operations in $\F$ for
the conversion from normal to power basis. This proves the first 
half of Theorem~\ref{thm:main2}.

%%%%%%%%%%%%%%%%%%%%%%%%%%%%%%%%%%%%%%%%%%%%%%%%%%%%%%%%%%%%

\subsection{Power basis to normal basis.}

Now assume $u \in \K$ is given in the power basis. Still writing the
elements of $G$ as $g_1,\dots,g_n$, the goal is to find coefficients
$c_i$'s in $\F$ such that
$$ \sum_{i = 1}^{n} c_i g_i(\alpha) = u.$$ Starting from this
equality, for any element $g_j$ of $G$, we have
$$ \sum_{i = 1}^{n} c_i g_jg_i(\alpha) = g_j(u).$$
Then, if $\ell$ is a random $\F$-linear projection $\K\to\F$, we get 
$$\sum_{i = 1}^{n} c_i \ell(g_jg_i(\alpha)) = \ell(g_j(u)), \quad  1 \le j \le n.$$
Introducing 
$$u'= \sum_{i=1}^{n} c_i g_i^{-1} \in \F[G]$$
and writing as before
$$s_{\alpha,\ell}=\sum_{j=1}^{n} \ell(g_j(\alpha))g_j
\quad\text{and}\quad
s_{u,\ell} = \sum_{j=1}^{n}\ell(g_j(u)) g_j \quad\text{in~} \F[G],$$ the
$n$ equations above are equivalent to the equality
$s_{\alpha,\ell}\ u' = s_{u,\ell}$ in $\F[G]$.

We use the algorithm of Section~\ref{sec:osum} to compute both
$s_{\alpha,\ell}$ and $s_{u,\ell}$; this takes $\osumcosttilde$
operations in $\F$, for $G$ polycyclic. If $\alpha$ is normal,
$s_{\alpha,\ell}$ is a unit for a generic $\ell$. Then, if we further
assume that $G$ is either abelian or metacyclic, it suffices to apply
the division algorithms given in the previous section to recover $u'$,
and thus all coefficients $c_1,\dots,c_n$. In both cases, the runtime
of the division is negligible compared to the cost $\osumcosttilde$ of
the first step. Altogether, this finishes the proof of
Theorem~\ref{thm:main2}.


%%% Local Variables:
%%% mode: latex
%%% TeX-master: "NormalBasisCharZero"
%%% End:
